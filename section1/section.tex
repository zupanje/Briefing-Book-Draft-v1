% don't remove the folling lines, and edit the defintion of \main if needed
\documentclass[../report.tex]{subfiles}
\providecommand{\main}{..}
\IfEq{\jobname}{\currfilebase}{\AtEndDocument{\biblio}}{}
% until here

%To have the commments written on the PDF: keep this macro
\newcommand{\commentsinout}[1]{#1}
%To have the commments hidden in the PDF: keep this macro
%\newcommand{\commentsinout}[1]{}

% this is a macro so you can make comments. See example below.
\newcommand{\KE}[1]{\commentsinout{{\bf{\color{brown} [KE: #1]}}}} % Comments by K. Ellis
\newcommand{\BH}[1]{\commentsinout{{\bf{\color{cyan} [BH: #1]}}}} % Comments by B. Heinemann
\newcommand{\FM}[1]{\commentsinout{{\bf{\color{orange} [FM: #1]}}}} % Comments by F. Maltoni
\newcommand{\AN}[1]{\commentsinout{{\bf{\color{teal} [AN: #1]}}}} % Comments by A. Nisati

\begin{document}

\chapter{Executive summary}
\pagenumbering{arabic}
Start of the text.

%\chapter{Theoretical overview}
%\newpage
%\subfile{\main/testsection/section}
%\newpage
%\subfile{\main/ewksection/section}

%\chapter{Strong Interactions}
%\section{State-of-the-art}
%\section{Hadronic structure}
%\section{Hot and dense QCD}
%\section{Precision QCD}
%\section{QCD and other disciplines}
%\section{Overview for QCD}

\chapter{Flavour Physics}
\section{Introduction/Theory of Flavour}
\section{Light sector: spectrum below GeV (short-, mid- and long-term)}
\section{Heavy sector (short-, mid- and long-term)}
\section{Flavour and dark sectors (short-, mid- and long-term)}
\section{The CKM matrix elements: prospects}
\section{Conclusions}

\chapter{Neutrino Physics}
\section{Theoretical introduction}
\section{Present knowledge of neutrino mixing parameters}
\section{Measurements of neutrino oscillation parameters}
\section{Determination of neutrino mass and nature}
\section{Search for new neutrino states}

\chapter{Cosmic messengers}
\section{Ultra High Energy charged particles}
\section{Ultra High Energy neutrinos}
\section{Gravitational waves}
\section{Multi-messenger astroparticle physics}
\section{Synergies with HEP}

\chapter{Beyond the Standard Model}
\section{Introduction}
\section{Electroweak Symmetry Breaking and New Resonances}
\section{Supersymmetry}
\section{Extended Higgs Sectors and High-Energy Flavour Dynamics}
\section{Dark Matter}
\section{Feebly Interacting Particles}
\section{Summary and Conclusions}

\chapter{Dark Matter and Dark Sector}
\section{Introduction}
\section{Dark matter and Dark sectors at colliders}
\section{Dark matter and Dark sectors at beam dump and fixed target experiments}
\section{Axions and ALPs}
\section{Conclusion}

\chapter{Accelerator Science and Technology}
\section{Accelerator technology, role of other regions, strategies for after LHC, or after Higgs factory}
\section{Present state of accelerator technology for HEP}
\section{Technologies for electroweak sector}
\section{Path towards highest energies}
\section{Muon Colliders}
\section{Plasma acceleration}
\section{Accelerators Beyond Colliders}
\section{Energy management}
\section{Complementarities and synergies with other fields}

\chapter{Instrumentation and Computing}
\section{Particle physics instrumentation}
\section{Computing and software for particle physics}
\section{Interplay between instrumentation and computing}
\section{Developing and preserving knowledge and expertise}
\section{Summary of key points}

\chapter{Synergies}

\end{document}
