% don't remove the folling lines, and edit the defintion of \main if needed
\documentclass[../report.tex]{subfiles}
\providecommand{\main}{..}
\IfEq{\jobname}{\currfilebase}{\AtEndDocument{\biblio}}{}
% until here

\begin{document}
% this is the extra information to be used for the general sections.

\def\circa#1{\,\raise.3ex\hbox{$#1$\kern-.75em\lower1ex\hbox{$\sim$}}\,}

\section{Summary and conclusions}
\label{sec:BSM-Conclusions}

In recent years, the scene of BSM research has been evolving rapidly, thanks to a wealth of new experimental data in particle and astroparticle physics. On the theoretical front, less emphasis has been given to unified frameworks able to deal simultaneously with many key questions in particle physics, and more attention has been given to models that address individual shortcomings of the SM or simply single unexplained facts. This has created a more fragmented landscape of research activity, where there is no single dominating trend, but multiple approaches pursuing different directions. The need to look for new theoretical paradigms is making today's research in particle physics very exciting, rich with opportunities for alternative and revolutionary ideas. In this situation, more than ever, an intense and diversified programme of new experimental projects is needed to unravel the many mysteries left unresolved by the SM and to provide clues for progress in theoretical speculations.

The current report reflects broadly the present state of the field. Instead of giving a comprehensive account of all BSM model variations and their phenomenological signatures, the analysis has focused on a representative set of cases that allow for an informative comparison of the reach of future experimental projects. At the beginning of the ESPP physics activities, four fundamental questions that would serve as a leitmotif for the BSM studies were identified and presented to the physics community at the ESPP Symposium in Granada. This chapter is concluded with a presentation, in the form of a summary, of those questions and the answers that have emerged from the  study.

\medskip
\noindent{\bf 1.~~To what extent can we tell whether the Higgs boson is fundamental or composite?}

\noindent Undoubtedly the Higgs boson is the centrepiece of today's BSM physics. Its discovery has led to an unprecedented situation in physics, since no fundamental scalar particles and no fundamental forces different from gauge forces had ever been observed prior to the Higgs. These facts are not mere curiosities, but are at the core of the main puzzles confronting particle physics today. 
Progress with these issues requires an experimental programme targeted at precision measurements of Higgs interactions and EW observables. This programme is a {\it clear priority for the future of particle physics}. Higgs precision measurements are especially efficient in testing strongly-interacting EW breaking sectors (such as in composite Higgs models), theories for EW breaking in which there are no weak-scale coloured particles associated with the Higgs (such as Neutral Naturalness), and theories in which the Higgs is mixed with other scalar states.
%Much can be learned from this programme about the dynamics responsible for EW breaking (see Sect.~\ref{chapHiggs}). %For instance, our present knowledge of the Higgs is not sufficient to exclude an EW phase transition in the early universe drastically different from the smooth crossover predicted by the SM. An important goal of the Higgs precision program is to gain the needed experimental information, especially from measurements of the Higgs self-interaction, to reconstruct the dynamics of the EW phase transition.

A central question for the precision programme is the nature of the Higgs boson, i.e. whether it is a fundamental or composite particle. Theories like SUSY suggest that the Higgs boson is as fundamental as any other SM particle, while models based on approximate Goldstone symmetries suggest that the Higgs has a composite structure, much like the pion in QCD. As shown in Sect.~\ref{sec:BSM-EWKNR}, this question can be quantitatively addressed by future colliders, which can test the `size' of the Higgs up to inverse distances $1/\ell_H \sim 10-20$~TeV, more than four orders of magnitude below the size of a proton. To put this result in perspective, we define the degree of compositeness $\delta$ of a particle with mass $m$ as the ratio between its effective size and its Compton wavelength $\lambda_C = 2\pi \hbar /m c$ (which is a measure of the particle's quantum nature). For a proton, which is a fully composite object, one finds $\delta_p \approx m_p/(2\pi \Lambda_{\rm QCD})\approx 1$. For a pion, which is a composite particle but emerges as a Goldstone boson below the QCD scale, one finds $\delta_\pi \approx m_\pi /(2\pi m_\rho ) =0.03$. Future colliders will be able to probe the Higgs degree of compositeness at the level of $10^{-3}$. Knowledge about the fundamental nature of the Higgs will give us decisive indications on the directions to pursue in future BSM research.

\medskip
\noindent{\bf 2.~~Are there new interactions or new particles around or above the electroweak scale?}

\noindent All dynamical frameworks that address the open problems of EW symmetry breaking predict that the Higgs boson must be accompanied by new particles or new phenomena. This conclusion is quite generic. In SUSY, the Higgs is elevated to a supermultiplet and, furthermore, the scalar structure is doubled. Composite Higgs, Little Higgs, neutral naturalness predict accompanying particles as well. Any dynamical explanation of Higgs naturalness requires partners to the top quark and gauge bosons with appropriate properties. Therefore, the hunt for new high-energy phenomena is an essential route towards progress in our understanding of particle physics. Future colliders are superb explorers of this route, as discussed in Sects.~\ref{sec:BSM-EWKNR}, \ref{sec:BSM-SUSY} and \ref{sec:BSM-ExtendedScalars}. The exploration of the high-energy frontier and the search for new heavy particles is another {\it clear priority for the future of particle physics}. Probing the mysteries of Nature at the smallest distance scales remains one of the most fascinating challenges in science. High-energy searches are especially efficient in testing theories for EW breaking with new coloured particles (such as SUSY) and a great variety of new phenomena not necessarily associated with the Higgs (such as additional gauge bosons and scalars, heavy resonances, or exotic particles such as leptoquarks). The power of the high-energy frontier lies in its versatility for exploring the unknown, but also extends to the programme of precision measurements (especially for energy-growing effects) and to searches for rare processes that benefit from high luminosity. 

The collider exploration of short distances can proceed through {\it direct} or {\it indirect} searches. Proposed future colliders can explore new physics extensively, up to scales of tens of TeV through {\it direct} searches. The direct exploration with colliders operating at higher energies is the only way to have hands-on access to new phenomena and to inspect their microscopic nature. As shown in this chapter, a variety of new-physics scenarios can be effectively tested in this way. For example (see Sect.~\ref{sec:BSM-SUSY}), direct searches translate into a probe of the degree of naturalness of SUSY theories down to a level of $10^{-5}$, testing deeply one of the guiding principles of particle physics. %We also note that the direct exploration of the tens-of-TeV region can be an important target for at least two reasons. First, some theoretical frameworks (such as neutral naturalness) are able to delay the onset of new coloured particles beyond the EW scale, but no further than about 10 TeV (without a tuning exceeding the percent level). The second reason comes from flavour constraints, which bound the scale of new flavour physics $\Lambda_{\rm F}$ to be above tens of TeV or so. The gap between the natural expectation for new physics in the EW breaking sector ($\Lambda_{\rm EW}\approx 100$~GeV) and the bounds on the flavour scale ($\Lambda_{\rm F}\circa{>} 10$~TeV) has always been a stumbling block for the construction of BSM models. Making the two scales coincide ($\Lambda_{\rm EW}=\Lambda_{\rm F} \approx 10$~TeV) would offer a change of perspective, possibly opening up new scenarios connecting EW and flavour physics.

An alternative experimental strategy is based on {\it indirect} searches. A particularly interesting class of indirect probes are tests of accidental symmetries or cancellation mechanisms in the SM, especially in the flavour sector (see Chapter~\ref{chap:flav}).
A second class of indirect probes are Higgs precision measurements (see Chapter~\ref{chap:ew}). Quite generically, modifications of Higgs couplings are proportional to the degree of fine-tuning of the theory, with a fully natural theory predicting ${\mathcal O}(1)$ effects in Higgs couplings. This link provides the basis for a comparison between the relative effectiveness of direct versus indirect searches. The third class of indirect probes are EW precision measurements (see Chapter~\ref{chap:ew}). The expected intensity and accuracy of future lepton colliders at the $Z$ and $WW$ thresholds will allow for improvements of indirect sensitivity of several orders of magnitude. The distinction between EW and Higgs precision tests is purely historical, as they are really two aspects of the same question. 

The fourth class of indirect probes is the study of deviations in SM scattering processes (examples are the contact interactions studied in Sect.~\ref{sec:BSM-EWKNR}). The important difference of these observables, with respect to the other three classes of indirect tests, is that their effects grow with powers of $s/\Lambda^2$ and therefore these measurements benefit not only from the statistics of high luminosity, but also from gains in the collider energy. For this reason, indirect tests are not only the domain of lepton colliders. Hadron colliders with very high luminosities are also effective and complementary in this respect, in particular when looking for processes that grow with energy or for rare processes, for which the large production rates at hadron colliders are essential.

The complementarity between direct and indirect searches can be illustrated with the example of a new resonance with mass $M$ and couplings $g_{Z^\prime}$ to SM particles (see Fig.~\ref{fig:Universal_Zp}). With {\it direct} searches, high-energy colliders can explore larger $M$ by increasing $\sqrt{s}$, and smaller $g_{Z^\prime}$ by increasing the luminosity. Virtual effects of the resonance can be detected {\it indirectly} by measuring deviations from SM predictions in the high-energy tails of distributions. These measurements can probe masses beyond the collider kinematic limit, but are sensitive only to the ratio $g_{Z^\prime}/M$. Higher energies allow for more effective probes of the ratio $g_{Z^\prime}/M$. This example shows the complementarity between the two experimental strategies, with {\it direct} searches being in general more effective in the weakly-coupled regime (i.e. small $g_{Z^\prime}$) and {\it indirect} searches in the strongly-coupled regime (i.e. large $g_{Z^\prime}$). Although this distinction provides a good general guideline, a precise comparison between the two strategies can only be performed on a model-by-model basis. 

{\it Indirect} searches have the advantage of probing particle masses well beyond the collider kinematical limit, but cannot identify the specific source of new physics.
Only discoveries in {\it direct} searches can give firsthand access to the microscopic structure of new phenomena. However, any new discovery, irrespective of whether it stems from direct or indirect searches, will certainly motivate a scientific programme of dedicated precision measurements. This was the case for the LEP programme which followed the discovery of the $W$ and $Z$ bosons, and it is today the case for a Higgs factory which is proposed to follow after the discovery of the Higgs boson.

In this report, strong emphasis has been placed on the interplay between direct and indirect searches at colliders in the exploration of the high-energy frontier. This interplay allowed us to make quantitative comparisons between the exploratory power of completely different experimental projects and to show the great complementarity of the two methods in the search for new physics. 

\medskip
\noindent{\bf 3.~~What cases of thermal-relic particles are still unprobed and can be fully covered by future collider searches?}

\noindent Dark Matter provides a fascinating link between large-scale astronomy and short-distance particle physics (see Chapter~\ref{chap:dm}). While the realm of possibilities for DM is still enormous, there are well-defined windows where particle physics can contribute in a unique way. An interesting prototype for DM is a heavy weakly-interacting particle. In Sect.~\ref{sec:BSM-DM} it has been shown how future colliders can test this hypothesis, demonstrating that the early-universe thermal origin of DM in the form of an EW doublet or triplet can be conclusively proven or ruled out. Other benchmark models for DM can also be effectively tested at future colliders and examples have been provided.

Other forms of DM, constituted of much lighter particles, can also be explored by particle-physics experiments (see Chapter~\ref{chap:dm}). With a variety of collider-based, beam-dump and fixed-target experiments, it is possible to probe the existence of new light particles (such as axions, ALPs, sterile neutrinos, etc.) which are potential candidates for DM, as well as being motivated by other particle-physics or astrophysics considerations (see Sect.~\ref{sec:BSM-FIPs}).

\medskip
\noindent{\bf 4.~~To what extent can current or future accelerators probe feebly-interacting sectors?}

\noindent The absence, so far, of unambiguous signals of new physics from direct searches at the LHC, indirect searches in flavour physics and direct DM detection experiments invigorates the need for broadening the experimental effort in the quest for new physics and in exploring ranges of interaction strengths and masses different from what is already covered by existing or planned projects. While exploration of the high-mass frontier remains an essential target, other research directions have valid theoretical motivations and deserve equal attention. Feebly-interacting particles (see Sect.~\ref{sec:BSM-FIPs}) represent an alternative paradigm with respect to the traditional BSM physics explored at the LHC. The full investigation of this paradigm over a large range of couplings and masses requires a great variety of experimental facilities. In this context, the physics reach of experiments at future colliders is complemented by beam-dump facilities which typically cover the range of low masses and extremely feeble couplings.

%A good illustration of today's multifaceted research activity in particle physics are feebly-interacting sectors (see Sect.~\ref{sec:BSM-FIPs}). While exploration of the high-mass frontier remains an essential target, many other research directions deserve equal attention. We simply do not know where new physics may hide and experimental exploration should leave no stone unturned. Feebly-interacting particles offer an alternative perspective to conventional approaches and could be the first manifestation of new hidden structures in nature. Their existence can be justified by theoretical motivations, as they give new solutions to traditional problems in particle physics, astrophysics and cosmology. However, the experimental search for these particles is not confined to theoretically motivated cases, but it is largely driven by the pioneering spirit of scientific exploration. Good coverage of the various forms of feebly-interacting particles requires a varied physics programme with different experimental techniques, ranging from high-energy colliders to beam-dump facilities, as these techniques turn out to be complementary in their search sensitivities.

\end{document}
