
 Since the last update of the European Strategy, a plethora of new experimental results  has been achieved in flavour physics.  No indisputable evidence of new physics has  emerged so far,  though. 
 The rationale for the observed pattern of masses and mixings of quarks and leptons thus still remains a fundamental open question, which calls for new physics laws.  Precision flavour physics is a fundamental tool to discover them.


  The probing power of flavour physics is manifest from the comparative effective analysis of Fig.~\ref{fig:NPscales}. In the near future, the sensitivities of several observables will reach very high NP scales, $10^2-10^5$~TeV -- scales which are beyond the reach of high-energy colliders.  Note that this analysis does not claim that physics at --or below-- the  scale depicted is guaranteed to provide a measurable deviation from the SM.  All we claim is that a new physics scale  might be first signalled via 
flavour measurements. In spite of its 
 caveats, the figure illustrates well the great physics reach of flavour observables versus that of  direct and electroweak precision searches. Overall, these three quests are complementary and essential.  

 

The field of flavour has been traditionally explored through a wide spectrum of experiments, ranging from low- and intermediate-energies, such as  EDMs, dedicated muon, kaon, tau, charm and beauty experiments, to the high-energy frontier (the LHC experiments), all the way to feebly interacting particle searches. Measurements by these diverse experiments and facilities lead to valuable complementary information on the different pieces required to assemble the flavour physics mosaic. 

In the \textbf{short- and mid-term}, the expected progress of running experiments, the sensitivity goals of those already foreseen and the proposed upgrades of existing ones, will enable to determine a wide range of flavour observables with unprecedented precision (or establish new impressive limits).

Searches for EDMs of various types of particles are currently being performed. 
Substantial improvements in the short and mid-term are expected: the neutron and proton EDMs down to $10^{-29} $ e$\cdot$cm, and  $10^{-30} $ e$\cdot$cm  for the electron. Dedicated searches for charged lepton flavour violation in the muon sector foresee nominal improvements by as much as a factor of $10\,000$,  owing to the  high-intensity muon beam programmes at PSI, FNAL and J-PARC.   Furthermore, current hints of lepton non-universality in both charged-current and neutral-current semileptonic $B$ decays are expected to be  unambiguously tested.


In the quark flavour arena, kaon physics proceeds with a steady pace of improvements: efforts towards a substantial improvement on sensitivity for the $K^+ \to \pi^+ \nu \bar \nu$ decay 
and a first measurement of the $K_L \to \pi^0 \nu \bar \nu$ decay are underway.  The upcoming results expected from NA62 and the evolution of the Japanese project will guide the future European steps in this research field. 
Heavy quarks, and in general heavy flavour physics will continue to play an important role also in the post-LHC era. Furthermore, the improvements in lattice-QCD calculations are expected to keep pace with advances in  experimental precision,  
 motivating better measurements of observables critical to test the CKM paradigm.  A flourishing new area is also opening in the mid-term for the exploration of NP in flavourful Higgs couplings, as well as in top and gauge boson interactions. 

The LHCb Upgrade II would allow 
 for the full exploitation of the flavour physics potential of \HLLHC  and increase the explored NP mass scale by close to a factor two with respect to the short-term LHCb Upgrade~I.
It would also provide a bridge towards larger scale collider facilities.
Analogous considerations are valid for a possible upgrade of Belle II (Belle III).
Essential to the success of the physics programme is the experiment capability of charge particle identification and systematic uncertainty control
in different  environments 
($pp$  and $e^+e^-$).


In the \textbf{long-term}, experiments at a future high-luminosity $e^+e^-$ collider 
would perform unique heavy-flavour studies 
in specific channels. 
Circular colliders operating at the $Z$  
pole (in particular \FCCee) 
can strongly contribute to develop searches for  the charm Yukawa coupling and flavour-violating Higgs and $Z$ couplings, lepton flavour violation and precision tau physics, and dark sector searches.   
At a circular high-energy $pp$ collider, the top Yukawa coupling, flavour-violating top-Higgs couplings,  and other heavy flavour physics program related to $b$ and $c$ quark decays would be best studied with a dedicated experiment, along the lines of LHCb.



Furthermore,  from both the experimental and the theory side, a novel synergy  between the searches for flavour violating decays and that  for feebly interacting and dark particles is emerging. Searching for exotic signatures in flavour violating decays  
may have profound implications for our understanding of the Universe, and should be part of any broad program of searches for dark sectors. 
 High energy colliders will explore a large number of signatures and cover a large fraction of the parameter space for the high-mass range (above 10~GeV). 
Nevertheless fixed target smaller scale experiments, LHC projects dedicated
to long-lived particles and beam dump facilities may provide complementary information to explore a lower mass range  
(1~MeV - 10~GeV) and open new interesting research lines. 

\bigskip
The foreseen unprecedented improvements in sensitivity, together with the novel  channels to be explored, will make  the next decade particularly exciting for the flavour physics arena. 
In summary, the combination of
quark and lepton searches 
for flavour and CP violation  
 at different frontiers is 
a formidable tool to discover new physics. 
Flavour physics must be a crucial ingredient of the future strategy of particle physics. 


