
%This is the text of the theory introduction
%Contents: mass terms, oscillations, nature of neutrino, new states, NSI

%The discovery of neutrino oscillations proves that at least two neutrinos have non-vanishing masses and there is non-trivial mixing between mass and flavour states in the lepton sector. 

Thanks to the recent discoveries in the sector of neutrino oscillations
 we have now a clear zeroth-order picture of neutrino properties, which however, raises a number of theoretical and phenomenological questions: Why are neutrino masses many orders of magnitude smaller than any other fermion mass in the Standard Model? Are neutrinos their own antiparticles? What are the actual values of neutrino masses (absolute mass scale and mass ordering)? Is the CP symmetry violated in lepton mixing? What are the precise values of the mixing angles and why is lepton mixing so much different than quark mixing? Are there observable deviations from the standard three-neutrino picture (e.g., non-standard interactions or non-unitarity of the mixing matrix)? 
Answering these questions is the main focus of the present and future neutrino experimental program. It is of paramount importance as it offers a unique window on the physics beyond the Standard Model. Furthermore, the hypothesis of leptogenesis for the generation of the baryon asymmetry of the Universe links the origin of neutrino mass with the origin of matter.

The mechanism behind neutrino mass is unknown. 
%The gauge symmetries and the field content of the Standard Model do not allow neutrino masses at the renormalizable level. Therefore, some extension of the Standard Model is necessary. 
To obtain finite neutrino masses, the Standard Model has to be extended in some way.
A minimal extension is to introduce gauge-singlet neutrinos (so-called right-handed or sterile neutrinos) which would allow to write down a Dirac mass term for neutrinos, in the same way as for all other fermions. This could indeed be the only source of neutrino masses, but in this case coupling constants need to be smaller than $10^{-11}$ and lepton-number conservation has to be postulated as a fundamental symmetry. However, the electric charge-neutrality of neutrinos offers also the possibility of a Majorana mass term, which would imply that neutrinos are their own antiparticles and break lepton-number by two units. Many possibilities for generating Majorana neutrino masses are known, including the seesaw mechanism with right-handed neutrinos and/or with an extended scalar sector, or radiative neutrino mass models. Therefore, the search for lepton-number violation, in particular via neutrinoless double-beta decay, addresses a fundamental property of the theory of neutrino masses. 

Neutrino masses by themselves do not provide guidance towards the energy scale of new physics responsible for generating them. There is a vast range for the scale of new physics extending from sub-eV up to the GUT scale of $10^{16}$~GeV. In order to make progress in view of this multitude of possibilities a wide range of complementary observables needs to be explored. These include (i) the search for sterile neutrinos at various different mass scales including oscillations at the eV scale and heavy neutral leptons at collider and beam dump experiments, (ii) lepton number violation in neutrinoless double-beta decay or at high-energy colliders, (iii)~charged lepton-flavour violation, (iv)~precision measurements in the neutrino sector, and (v)~search for non-standard neutrino properties such as exotic interactions or non-unitarity of the \mbox{3 $\times$ 3} mixing matrix.

Since new states or new BSM interactions are required to explain the neutrino masses, the experiments must be ready for unexpected phenomena, and if possible provide alternative complementary measurements to over-constrain the three-flavour parameters. An example of these phenomena is given by Non Standard Interactions (NSI) between neutrinos and the other fermions. NSI are today constrained only weakly and they might modify the propagation of neutrinos in matter. 


