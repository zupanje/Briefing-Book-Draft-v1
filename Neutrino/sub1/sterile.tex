
%These questions are in part motivated by the so-called LSND anomaly, 
The hypothesis of sterile neutrinos is in part motivated by the LSND, MiniBooNE,
Reactor, and Gallium anomalies: these are results from short baseline accelerator based, reactor based, and source based experiments that cannot be interpreted in terms of 3 neutrino mixing.  
However there is strong tension between electron neutrino appearance and muon neutrino disappearance results and even with several new neutrino states
sterile neutrino oscillations
are unable to explain all the experimental data.
%For the accelerator based experiments, the LSND experiment reported in the late 1990s, a 3.8$\sigma$ excess of anti-electron neutrino events in a Decay at Rest pion beam.  The MiniBooNE experiment followed on this result at higher energies but longer baseline, so the same $\frac{L}{E}$ , observing a low energy excess of electromagnetic events, compatible with LSND in models involving one or more sterile neutrinos.  For the reactor based experiments, new calculations of the predicted reactor flux, motivated by experiments measuring $\theta_{13}$, called into question interpretations of reactor meausrements spanning 1956-2003.  With the newly predicted reactor flux, these experiments now showed a 2-4$\sigma$ deficit of anti-electron neutrino events which could be consistent with short baseline oscillations.  For the source based experiments, results from calibration runs from the GALLEX and SAGE experiments suggested a deficit of electron neutrino interactions, also consistent with disappearance.  
Each of these anomalies is at the 2--4\,$\sigma$ level: they motivated a program of  new experiments to understand the nature of the anomalies.  

A host of reactor based experiments are now underway at both power and research reactors to look for anti-electron neutrino disappearance.  %Many of these take advantage of movable detectors to diminish reliance on the predicted flux and look for results as a function of baseline.  Some take advantage of overburden to reduce backgrounds and others of clever reconstruction techniques to do the same.   
The experiments NEOS (South Korea), DANSS (Russia), Neutrino-4 (Russia), SoLiD (Belgium), STEREO (France), and Prospect (US), have presented first results which provide hints and limits, and are in continued running for more definitive results.  Some of these experiments, for example the Prospect experiment, are also performing measurements which will inform and constrain the prediction for the un-oscillated reactor flux.   

A series of accelerator based experiments~[ID137] %\cite{sblfnal-esppu} 
at Fermilab in the US are addressing the accelerator based anomalies. The MicroBooNE experiment uses the same neutrino beam as MiniBooNE, but a different detector technology, Liquid Argon Time Projection Chambers, to address the low energy excess observed by MiniBooNE.  While MicroBooNE is running and first results are anticipated soon, the SBND and ICARUS experiments, representing the Phase 2 of this program, are under construction and installation.  SBND serves as a near detector to measure the un-oscillated flux and ICARUS as a large mass far detector to fully address the LSND anomaly. ICARUS has benefited from CERN support (WA104, NP01) to move the existing detector from LNGS to CERN, to overhaul it, and then to move it to Fermilab. 

In addition to these dedicated experiments, there are a number of other experiments which can help understand these anomalies.  These include the MINOS+ experiment at Fermilab, IceCube (South Pole), the KATRIN (Germany) and $^{163}$Ho neutrino mass experiments, and results from cosmology.  The BEST source experiment is in the planning stage and the JSNS2 experiment, a direct test of LSND, is about to start running in J-PARC (Japan).  Results from these experiments in the upcoming years should help to clarify the situation, be it in terms of new physics or of other explanations.
