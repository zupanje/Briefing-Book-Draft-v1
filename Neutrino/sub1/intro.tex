The discovery of neutrino oscillation proves that neutrino have non-zero masses. This is one of the few solid experimental proofs of physics beyond the Standard Model, as new interactions or new elementary particle states are needed to introduce this mass term in the Lagrangian.

Moreover, the extremely low mass of the neutrinos, well below the eV scale, sets them far apart from the other fermions. This extraordinary lightness might be related to new physics at a very high scale, as proposed by the see-saw models. There is the tantalizing possibility suggested by the leptogenesis hypothesis that the phenomena at these high scales could explain the baryon asymmetry in the Universe. Moreover neutrinos could be a completely new kind of particle, a Majorana fermion, identical to its antiparticle. If this is realized in nature, new processes violating the conservation of the lepton number are possible.
For all these reasons, neutrinos are therefore widely considered as a unique window to BSM physics.

These considerations have triggered a very vibrant experimental program world-wide that has made rapid progress in the last years. With the discovery of the third mixing angle $\theta_{13}$ the three neutrino mixing framework has been established.

A new exciting phase of experiments and discoveries opens up, that will be covered in this chapter. Moreover, as neutrinos are also a special probe of dense astrophysical systems, there is a strong synergy at many levels with astroparticle physics that will be covered in Chapter~\ref{chap:cosm}.


