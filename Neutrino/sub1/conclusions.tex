
The study of the neutrino nature and properties, motivated by the unique window that this particle offers on BSM physics, is entering a new phase with a rich array of experiments using neutrinos from accelerators but also from other sources (atmospheric, solar, reactor, cosmic neutrinos).

The first priority is today on the completion of the program of measurements of the oscillation parameters, most notably the CP-violating phase of the mixing matrix and the neutrino mass ordering. 
Two strong and complementary experimental programs are in preparation towards this goal in USA and Japan with the DUNE and Hyper-Kamiokande experiments. Following the recommendations of the 2013 European strategy, there is a strong participation of European physicists in both programs, benefiting of CERN support, most notably through the CERN Neutrino Platform. This approach receives today the support of the neutrino community at large as shown by the recent Neutrino Town Meeting in 2018 and its conclusions~[ID45]. A balanced support to this world-wide effort from Europe will allow to secure the determination of the oscillation parameters, aim at the discovery of CP violation and test for possible deviations from the three-neutrino framework.

To extract the most physics out of DUNE and Hyper-Kamiokande, a complementary program of precision supporting measurements is needed. 
NA61 and its upgrade are an important component of this program for the determination of the neutrino fluxes. A study should be set up to evaluate the possible implementation and impact of a facility (based on the ENUBET or $\nu$STORM concepts) to measure the neutrino cross-sections at the \% level.

Other important complementary experiments are in preparation in China (JUNO) and in Europe with the KM3NeT/ORCA program, using reactor and atmospheric neutrinos respectively. They have the potential to discover the mass ordering and to do other precision oscillation measurements.

The study of the neutrino absolute mass and nature (Dirac or Majorana) is the other priority in the field, covered by both laboratory and cosmology measurements.
%Moreover, as neutrinos are also a special probe of dense astrophysical systems, there is a strong synergy with astroparticle physics that will be covered in chapter~\ref{chap:cosm}. 
