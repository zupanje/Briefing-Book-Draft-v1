\section{Interplay between instrumentation and computing}
The requirement of efficiency to extract the maximum physics potential from the available computing resources requires an increasingly "holistic" approach to the design of experiments and their associated computing and software systems. 

For example, the detector design decisions can either ease or place a huge burden on the offline computing and simulation. Full simulation studies are required not only to assess the level of background and optimize the detector performance,  but also to understand the computational costs in simulation and reconstruction arising from decisions on  the geometry, segmentation and situation of detectors. Subsequent evaluation of the detector design must include the computing burden as a metric. 

The online processing has an important role in a holistic approach. Many submissions illustrate the trend towards moving more complex algorithmic processing into the  online systems [\BV{need to add citation to submissions here}]. This is in parallel with an increasing trend to monolithic devices that integrate TDAQ functions into the devices as it reduces the offline burdens. Experiment design should consider the use of detector electronics to do  processing and data reduction at an early stage. In addition, offline-like reconstruction is increasingly possible in online triggering systems, with further reductions in the offline data volume without loss of physics performance.

There is also a need for a more holistic approach within the computing and software systems. For example, the design of the overall computing system must take into account both the hardware costs and the costs of operation. Equally, there is a trade off between effort in the development of efficient software and event models and consequent resource requirements. While these may fall into different accounting categories (e.g. recurrent versus capital costs), they need to be tensioned. This in turn necessitates realistic estimates of required software development effort.

Organizationally, computing and detector R\&D have had little cross-talk. Indeed, computing and software has often been considered as a secondary activity after detector design. In future, it would be desirable that projects view computing and software responsibilities on an equal footing with sub-detector responsibilities. 


