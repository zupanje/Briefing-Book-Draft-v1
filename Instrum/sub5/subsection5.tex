\section{Developing and preserving knowledge and expertise}

%\BV{make sure to say somewhere that these issues were already identified for many years, e.g. in previous EPPSU, and unfortunately this is still the case now.}

The scientific questions tackled by the particle physics community are long term inquiries, and as such, in order to reach the community's scientific goals, human factors need to be carefully considered.  
One of the challenges faced by the particle physics community is an adequate level of development and preservation of expertise in instrumentation and computing-related R\&D activities.  For the benefit of the entire research field, it is of utmost importance that both types of activities be recognized correctly as fundamental research activities bearing a large impact on the final physics results.  This requires, now more than ever, a major change in paradigm in the community. 


\subsection{Recruitment}

For the future of our field, it is essential to attract brilliant young physicists to the interesting challenges of instrumentation and computing-related R\&D.  To effectively do so, these activities must be recognized not only as a means to allow scientists to do physics analyses, but as research areas requiring a high level of imagination and creativity.  Attractiveness of these research areas could also be increased by more effectively communicating to prospective students that innovative ideas in these areas will often have far reaching impact on industry and society.  Furthermore, in outreach and public relations, it would be highly beneficial and desirable that the community more systematically highlights the technological dimensions of particle physics.  Particle physics challenges have the potential to continue to attract the best and brightest students.


\subsection{Training}

The knowledge, specialization and expertise required in the present and future field of particle physics are extensive.  Students and postdocs often lack basic knowledge in detector technologies, electronics, mechanics, software and simulation.  One of the reasons is that these specializations are rapidly evolving and university courses become insufficient to prepare students in these technical aspects.  
In addition, excellence in instrumentation development is often not valued and sufficiently recognized at the universities so as to attract the interest of students in this branch of research.
As a consequence, it also becomes difficult to attract young people to work in R\&D for instrumentation and computing. Yet, paradoxically, detector (system) prototyping is an excellent fertile training ground for young particle physicists.

It would be profitable to enhance, already at the level of university training and/or degree requirements, the basic knowledge required for applied physics activities.
One way would be to set up mixed trainings for applied physicists and engineers, for example through platforms that can bring universities and small laboratories together. Such activities would increase the training classes and find root in the departments, by creating also a positive feedback in terms of increased opportunities for careers in academies. 
%The development of specialized particle physics degree(s) with particular focus on applied activities should be considered.
In this respect, the community could consider creating a document targeting universities and laboratories, advocating for the training specificity of instrumentation and detector development.
The means that universities and small laboratories have, however, are typically limited. It remains therefore extremely important to preserve initiatives of internships at CERN and other large laboratories.

With the long time scale often associated with particle physics experimental projects, opportunities for students to participate in all phases of an experiment are becoming more and more scarce.  Investment in the specialized education of young physicists in the form of schools in particle physics instrumentation and/or scientific computing (e.g.\ EDIT school) is highly beneficial.


\subsection{Expertise preservation and career opportunities}

Result from the 2018 ECFA survey~[ID68] of the community show that career perspectives for detector experts were perceived to exist in research, industry and a tertiary sector requiring advanced software development skills by 39\%, 66\% and 80\% of respondents.  This perception is driven by the reality of a flawed academic hiring model in which data analysis contributions are given more weight than instrumentation and/or computing-related research activities.  Whether it is the cause, or a consequence, of this hiring model, the reality is that the particle physics community has a strong tendency to under-sell the intellectual challenges---and satisfaction---of detector and computing-related R\&D work. 
Furthermore, compounded to this perception is the reality that except for a very few geniuses, individuals cannot typically be expert and innovate simultaneously in all areas of physics analysis, detectors, computing, teaching, outreach, etc.  As a result, few attractive career opportunities currently exist for individuals with specific instrumentation or computing expertise.   

The success and future of particle physics will directly depend on the community's ability to greatly improve in a systematic way the recognition and career opportunities of detector and computing experts.  One avenue to address this challenge is for the community to strengthen its effort to advocate for the recognition of these research areas in universities hiring plan, possibly exploiting synergies and potential interests with engineering and computing science department.  National and international laboratories also have the opportunity to play a major role in the community, for example, by creating prestigious career paths for physicists with particular expertise in instrumentation or computing-related research.  In doing so, care should be taken to avoid the development of a two-tier career system composed of ``scientific'' and ``applied'' streams with little or no cross-breeding, thereby only reinforcing the current {\it status quo}.  Other avenues to address this challenge could be to setup specialized and attractive grants for instrumentation and computing R\&D, as well as prizes recognizing both young and experienced scientists. 


%then issue recommendation for laboratories to create a specific and prestigious career path for applied physicists (e.g. setup grants explicitly for instrumentation and computing, prices for young people working in the technical areas, boost long-term positions)

%Education and expertise transfer necessary to maintain knowledge and capabilities

%first step: make community attentive and informed about the potential and the needs of the technical workforce

%It would be highly fruitful to encourage, with respect to the academy, a plan that allows equal-opportunity careers in instrumentation at the universities, for both students and professors.



