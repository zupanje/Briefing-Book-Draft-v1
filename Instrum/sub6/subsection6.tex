\newpage
\section{Summary of key points}

The following summarizes some of the key points discussed in this chapter.

\begin{itemize}
    \item It is critically important, more than ever, for the community to maintain a strong focus on instrumentation R\&D and to foster an environment that stimulates innovation, with the primary goal of addressing the well-defined technological challenges of future experimental programs. 
    
    \item Computing challenges are immediate and need to be addressed now through a vigorous R\&D program. A cross-computing view tensioning new hardware costs with the costs of software development is required.
    
    \item Both detector and computing development efforts benefit from the existence of networks and  consortia, within particle physics as well as extending to other disciplines and with industry.
    There is a clear need to strengthen existing R\&D collaborative structures, and create new ones, to address future experimental challenges of the field post HL-LHC.
    
%    and computing require continued and strong networks and consortia, both with other disciplines and with industry. The benefits are the sharing of experience, solutions and development costs, but also the strong links with industry have inherent benefits.
    
    \item It is becoming increasingly vital to take a more holistic approach to detector design which includes impact on computing resources; however, the detector and the computing/software communities have been drifting apart, and individuals that can bridge the growing gap are rare. This is a challenge to our community. 

    \item While producing experts who then go on to shape the wider world is a significant benefit to society,  %from the community, 
    a limited amount of success in attracting, developing and retaining instrumentation and computing experts poses a growing risk to the field of particle physics.  It is of utmost importance that both instrumentation and computing development activities be recognized correctly as fundamental research areas bearing a large impact on the final physics results.

%    Instrumentation and computing development activities need to be recognized as fundamental  research areas 

    
%    there are growing risks to  projects from the inability to developing detector and computing experts with the required skills base, and our inability to retain these experts, as the academic system does not currently reward such expertise. 
    
    
\end{itemize}
\vfill