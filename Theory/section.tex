%% don't remove the following lines, and edit the definition of \main if needed
\documentclass[../report.tex]{subfiles}
\providecommand{\main}{..}
\IfEq{\jobname}{\currfilebase}{\AtEndDocument{\biblio}}{}
% until here

%%%%%%%%%%%%%%%%%%%%%%%%%%%%%%%%%%%%%%%%%%%
\def\Eq#1{Eq.~(\ref{#1})}
\def\Fig#1{Fig.~(\ref{#1})}
\def\Sec#1{Sec.~(\ref{#1})}
\def\App#1{App.~(\ref{#1})}
\def\beq{\begin{equation}}
\def\eeq{\end{equation}}
\def\bea{\begin{eqnarray}}
\def\eea{\end{eqnarray}}
\def\eq#1{eq.~(\ref{#1})}
\def\mp{M_{\rm Pl}}
\newcommand{\MeV}{\,{\rm MeV}}
\newcommand{\GeV}{\,{\rm GeV}}
\newcommand{\TeV}{\,{\rm TeV}}
\def\hc{\rm h.c.}
\def\circa#1{\,\raise.3ex\hbox{$#1$\kern-.75em\lower1ex\hbox{$\sim$}}\,}
\newcommand{\xxx}[1]{{\color{blue}\sl#1}}
%%%%%%%%%%%%%%%%%%%%%%%%%%%%%%%%%%%%%%%%%%%
%\usepackage{hyperref}
%\usepackage{mathptmx}
%\usepackage{amsmath}
%\usepackage{xcolor}


\begin{document}
\baselineskip=14pt
\linenumbers
\chapter{Theoretical overview}
\label{chap:th}
\section*{The role of exploration in particle physics}

Exploration of the unknown is the main driver of fundamental science. 
  The goal of particle physics is to push the frontier of knowledge deep into the smallest fragments of spacetime and unravel the natural phenomena that occur at the most minute distance scales. This line of research has delivered some of the most extraordinary discoveries in science, which not only have revealed the inner workings of particle interactions but have truly revolutionised our understanding of the physical laws that govern the Universe. Those laws have allowed us to decipher the properties of the Universe at the largest distance scales and reconstruct its time evolution back to the earliest stages. This path of discoveries and knowledge has continued with the latest generation of experimental projects in particle physics, among which the LHC is the most prominent example. Although this broad research programme is still ongoing, particle physics is already planning the next stage of exploration. While research that culminated with the LHC has established the Standard Model (SM) as the successful description of particle interactions, we are still confronted with many unresolved puzzles and open problems that can be tackled only with a bold experimental programme and with substantial technological advances.
%When the goal is exploration of the unknown, by its very nature it is difficult (or impossible) to foretell the discoveries that an experimental project may encounter. 


\section*{Can we predict new discoveries?}

When the goal is exploration of the unknown, by its very nature it is difficult (or impossible) to foretell the discoveries that an experimental project may encounter. The value of an exploratory project should not be measured by the number of promised new discoveries, but by the importance of the questions addressed and by the amount of fundamental knowledge that can be extracted from its results.
There is a remarkable exception to the general rule concerning our inability to predict the unknown.
 It follows from a peculiar property of quantum field theories (QFT). A QFT can ``predict its own destruction,''\footnote{This expression was used by Pilar Hernandez in her talk at the Open Symposium on the Update of European Strategy for Particle Physics in Granada.} in the sense that from low-energy measurements alone one can infer the existence of new phenomena that must {\it necessarily} occur below a calculable high-energy scale $\Lambda$, even if the theory is unable to predict what these new phenomena are. It is of course a very privileged situation for experimental searches.  
 %because new discoveries below a given energy scale can be predicted with mathematical certainty. This 
 This was indeed the case of the SM without the addition of the Higgs, which predicted its own destruction at energies below a TeV, as subsequently confirmed by the LHC with the discovery of the Higgs boson. 
 
 Does the SM today, after the inclusion of the Higgs, predict its own destruction?  The answer is negative: unlike the circumstances at the start of the LHC, today particle physics is not in the (rare and special) situation to predict new discoveries  with mathematical certainty at or under an energy scale within reach.  
 Remarkably, the SM properties are just right to keep all its coupling constants under control, up to meaningful high energies. The Higgs quartic coupling evolves towards an instability, but this is reached at sufficiently high energy to ensure that the lifetime of the electroweak vacuum is much longer than the age of the Universe. This remarkable self-consistency of the SM is very sensitive to the values of the coupling constants and it would not hold if the couplings were only slightly different from what we observe.  In a different realm, as soon as gravity is included the SM does predict its own destruction at an energy scale of $10^{19}$~GeV or below. In addition, neutrino masses suggest the self-destruction of the SM in the neutrino sector somewhere below (and possibly much below) $10^{15}$~GeV. However, these upper bounds on the cutoff energy $\Lambda$ are too weak to guarantee discoveries with absolute certainty at foreseeable future colliders. The situation could suddenly change if the LHC, or any other current experimental project, found evidence for new 
 %non-renormalisable 
 interactions with low scale $\Lambda$.   

\section*{Open questions in particle physics}

%The conclusion from the previous discussion is that, unlike the circumstances at the start of the LHC, today particle physics is not in the (rare and special) situation to predict new discoveries with mathematical certainty. 
 What drives the field towards the next generation of experiments is the awareness that we are facing fundamental questions that can be addressed by the scientific method,  and whose answers will significantly enrich human knowledge. 
 In the following we present some of the open questions in particle physics today. 
  Their breadth clearly requires 
  %that advances in particle physics require 
  a diversified research programme with different experimental objectives and techniques, with bold projects pushing the energy and precision frontiers, and with substantial theoretical involvement. The open questions 
  %we describe 
  are not independent, but deeply interconnected. This reflects the maturity of our global understanding of the particle world and the strong links between all aspects of particle physics, reaching out to neighbouring fields like cosmology and astrophysics. 
  %The deep interconnections between the open questions also give a good indication that 
  Any new discovery is thus likely to affect our understanding of particle physics in multiple directions.

\medskip\noindent 
{\it 1~~~Electroweak Symmetry Breaking}

\smallskip \noindent
 With a ground-breaking result, the LHC has established the existence of the Higgs boson as the main agent of the spontaneous breaking of electroweak symmetry. In the context of the SM, all the parameters associated with the Higgs (scalar potential and couplings to gauge bosons and fermions) are 
 %predicted and 
 related to measured quantities. And yet, our understanding of the electroweak symmetry breaking dynamics is far from being satisfactory. The Higgs sector remains a conceptual mystery. 

The problem is related to the nature of the Higgs boson, which is an object different from any other particle we have encountered so far because, according to the SM, it is a fundamental particle with no spin. Contrary to particles that carry spin, for which the massless and massive cases are distinct as they correspond to different numbers of physical degrees of freedom, a massless spinless particle can be turned into a massive one without adding any new physical excitation. This property becomes lethal in the quantum world, since the mass of the Higgs boson becomes wildly sensitive to quantum fluctuations. Its spinless nature leads to another distinguishing feature: the existence of new types of interactions that are different from the gauge interactions that characterise the familiar four fundamental forces of nature. While the structure of gauge forces is restricted by the mathematical properties of symmetry, the new forces introduced by the Higgs boson are less constrained, and this leads to a large number of undetermined parameters. The Higgs alone requires the introduction of 15 new free parameters, as opposed to the strong and electroweak forces which are described by only 3 parameters. When compared with the structural simplicity of the gauge sector, the Higgs sector looks suspiciously provisional. Essentially all problems or unsatisfactory aspects of the SM are ultimately related to the structure of Higgs interactions. Our poor understanding of the Higgs sector at a deeper level, and its novelty in terms of physical properties, make Higgs precision measurements one of the most pressing issues of any future programme in particle physics.

The discovery of the Higgs boson has opened a new research programme, which is a clear priority for the future of particle physics. Precision measurements of Higgs properties enable us to study in depth the most puzzling sector of the SM, opening the door towards a deeper understanding of the mechanism for electroweak symmetry breaking. Future colliders promise an unprecedented scrutiny of the Higgs properties (see Chapter~\ref{chap:ew}). They can explore extensively the nature of the Higgs boson and the question of whether the Higgs is accompanied by other related spinless particles or not. Moreover, if the Higgs were a composite state rather than a fundamental particle as predicted by the SM, its size could be probed at future Higgs factories down to distances of $10^{-20}$~metres, about five orders of magnitude below the size of the proton (see Chapter~\ref{chap:bsm}). 

 The Higgs programme goes hand-in-hand with the programmes of electroweak precision measurements and flavour physics, which can probe the existence of new physics in a way complementary to direct searches (see Chapters~\ref{chap:ew} and \ref{chap:flav}). Moreover, very high-energy collisions offer the opportunity of studying the interplay between short and long-distance effects. 
%in cases such as electroweak resummation and gauge-boson showering. 
 This is a new environment in which one can test the infrared properties of non-Abelian gauge theories in a regime which, unlike the case of QCD, is fully perturbative.   


\medskip
\noindent {\it 2~~~Higgs Naturalness}

\smallskip
\noindent
A related puzzle in particle physics is the question of Higgs naturalness. 
% It stems from
 The problem arises because of the quantum sensitivity of the Higgs mass to possible new physics scales, while  various experimental measurements point
%, though, 
 towards a large separation of the latter and the Higgs mass ($\Lambda \gg m_h$).  The discovery of the Higgs boson has made the problem more concrete, and the lack of evidence for new physics has widened the gap between $\Lambda$ and $m_h$, making the tension more severe.
 %, for no apparent reason. 
 In the language of  Effective Field Theories (EFTs), naturalness arises by viewing EFT parameters as functions of more fundamental ones: any specific structure in the EFT, like the presence of a very small parameter, should be accounted for by symmetries and selection rules rather than by accidents. When the criterion is applied to the Higgs mass, one famously finds that $\Lambda \gg m_h$ is inconsistent with the predicate of naturalness.  Overall, the lack of novel signals suggests that the purely accidental symmetries of the SM appear to be mysteriously maintained up to the high energies tested at present, directly or indirectly. 

An interesting point of view 
%on this issue
 formulates the problem as arising from the clash between two concepts:
% intrinsic to EFTs:
 Infrared (IR) simplicity and Naturalness.\footnote{This formulation of the problem is due to Riccardo Rattazzi and the discussion here follows closely his talk at the Open Symposium on the Update of European Strategy for Particle Physics in Granada.} The SM enjoys IR simplicity in the sense that certain crucial experimental facts,  not mandated by the SM gauge structure, are nevertheless an expected consequence of global symmetries that emerge by pure accident because of the specific matter content found in Nature, e.g.\ approximate baryon and lepton number conservation, lightness of neutrinos, custodial symmetry, and suppression of flavour-changing neutral currents. The problem appears when the SM is embedded in a broader underlying framework. When the latter is probed at energies much below its fundamental scale $\Lambda$,  there is no generic reason for it to abide by the same accidental symmetries as the SM does. This would mean that IR simplicity in the SM can be obtained only at the price of a loss of naturalness in the formulation of the high-energy theory. Indeed, 
  %Another side of the problem is that 
  present models aiming to realize naturalness, such as supersymmetry or composite Higgs, invariably sacrifice simplicity. Those 
 %natural 
 extensions of the SM have concrete structural difficulties in reproducing the observed simplicity in flavour, CP violating and electroweak precision observables. In order to become phenomenologically viable, 
 %meet the corresponding experimental constraints, 
 %those scenarios 
 they must rely on artificial constructions mostly associated with ad-hoc symmetries, which in the SM are either not needed or automatic. In this perspective, the tension between simplicity and naturalness is what defines the problem.
 
  Alternatively, the concept of naturalness may appear in a different guise when applied to the electroweak sector. 
  %in the form we understand it today. 
  The increasing tension could well herald an exciting change of paradigm. 
 %  in history ultimatley lead to simplicity ultimately, as it has been repeatedly proven in history.
% Moreover, 
 In fact, the LHC results have already prompted theorists to broaden their perspective on the problem and pursue alternative solutions that may lead to unconventional experimental signatures. While much of this research is still ongoing, it is already clear that searches for new physics must take a broad approach. 
 
  In conclusion, Higgs naturalness remains a crucial open question. Whatever the underlying rationale,  more experimental investigation is mandatory to delimit and 
  %the issue of Higgs Naturalness and 
  clarify the issue. Understanding its role in the SM by probing its consequences at even higher energies will give us knowledge about the governing principles of Nature, and critical information for the future course of research in particle physics (see Chapter~\ref{chap:bsm}).


% IR Simplicity corresponds to approximately stripping down the theory to its renormalisable interactions only, when the theory is probed at energies much below its fundamental scale $\Lambda$. Because of various global symmetries that emerge accidentally in the SM at the level of renormalisable interactions, several crucial experimental facts like approximate baryon and lepton number conservation, lightness of neutrinos, custodial symmetry, and suppression of flavour-changing neutral currents remarkably and beautifully follow from IR Simplification. In other words, various experimental measurements point towards a large separation of scales between new physics and the Higgs mass ($\Lambda \gg m_h$) in the SM. Naturalness arises by viewing EFT parameters as functions of more fundamental ones: any specific structure in the EFT, like the presence of a very small parameter, should be accounted for by symmetries and selection rules rather than by accidents. When the criterion is applied to the Higgs mass, one famously finds that $\Lambda \gg m_h$ is inconsistent with the predicate of Naturalness. This means that IR Simplicity in the SM can be obtained only at the price of a loss of Naturalness, whenever the SM is embedded in an underlying framework. Another side of the problem is that models realising Naturalness, such as supersymmetry or composite Higgs, invariably sacrifice Simplicity. All these natural extensions of the SM have concrete structural difficulties in reproducing the observed simplicity in flavour, CP violating and electroweak precision observables. In order to meet the corresponding experimental constraints these scenarios must rely on artificial constructions mostly associated with ad hoc symmetries, which in the SM are either not needed or automatic. The tension between Simplicity and Naturalness is what defines the problem.

% The LHC results to date have shifted the balance towards Simplicity and made the question about the role of Naturalness more pressing. On one side, the discovery of the Higgs boson has made the problem more concrete. On the other side, the lack of evidence for new physics has widened the gap between $\Lambda$ and $m_h$, making the tension more severe. Only more experimental investigation can clarify the fate of the struggle between Naturalness and Simplicity. 

\medskip
\noindent {\it 3~~~Strong Interactions}

\smallskip
\noindent
Strong interactions play a central role in particle physics today, but the relevant questions are not about the validity of the theory or the search for its  possible extensions, because QCD gives a successful and satisfactory explanation of strong interactions. The questions are about how to relate QCD to long-distance phenomena (e.g.\ confinement), how to characterise the collective behaviours that emerge under extreme conditions (e.g.\ high temperature or high density), how to obtain reliable predictions in the non-perturbative regime (e.g.\ hadronic matrix elements) and precise predictions in the perturbative regime (e.g.\ higher-order calculations). Particularly challenging is the question of deriving from the first principles of QCD a description of phenomena at the interface between low and high energies. An example is understanding how fast-moving quarks and gluons cluster into colour-singlet hadrons. As well as being conceptually challenging, these questions are relevant in practice since they lead to a real limitation in the theoretical prediction of observables in hadronic collisions and flavour physics. 

QCD is the necessary tool for describing particle interactions with applications that range from heavy ions to proton collisions, from neutron stars to early-Universe cosmology. One of the most striking successes of the LHC has been to show that, by combining advanced data analysis and detector performance with refined theoretical QCD calculations, hadron colliders can perform measurements with previously unimaginable precision. This is an important legacy for future collider projects, which establishes proton colliders as precision machines. A number of the challenges listed above are of crucial importance for the full exploitation of the physics potential of present and future colliders,  both in searches for new phenomena and in performing precision measurements (see Chapter~\ref{chap:si}).

\medskip
\noindent {\it 4~~~Flavour Physics}

\smallskip
\noindent
The pattern of quark and lepton masses and mixings is one of the most puzzling open questions in particle physics, directly connected with the Higgs since its couplings to fermions are at the heart of the problem.
Generations of dedicated experiments have provided us with precision measurements of the corresponding parameters, revealing a pattern which has a highly non-generic structure and suggests an underlying organising principle. However, the origin of this structure and the nature of the organising principle remain mysterious. 

There is another reason that makes experimental exploration of the flavour sector particularly important. Because of special accidental symmetries and structural aspects (like the GIM mechanism), the SM predicts strong suppressions of certain  flavour-changing transitions. These suppressions have been confirmed experimentally. However, the suppression mechanisms in the SM are fragile and any small deformation of the theory can drastically change the predictions for flavour-changing processes. This property makes the study of rare flavour processes one of the most powerful probes of new physics, in some cases testing scales up to $10^5$ or even $10^6$ GeV.  Experimental hints for deviations from SM predictions in flavour processes are one of our best hopes to direct research towards the right energy scale where new physics may lurk. Dark matter itself may have flavour-violating interactions and an understanding of its structure would require advances in interdisciplinary explorations. Furthermore, flavour experiments are often sensitive to new light particles, possibly related to dark matter.

To the class of flavour processes belong $B$, $K$, $D$ meson and $\tau$ lepton decays, rare muon transitions, anomalous magnetic moments and electric dipole moments (EDM). Future experimental projects will be able to push further the exploration on all these fronts (see Chapter~\ref{chap:flav}), providing us with new fundamental knowledge about the particle world. In particular, testing new sources of CP violation in EDM is a powerful probe of theories beyond the SM, and may turn out to be a decisive tool to test hypothetical mechanisms for generating the observed cosmic baryon asymmetry.
% with weak-scale physics. 
Finally, a true understanding of the flavour puzzle must encompass both the quark and the lepton sector including neutrinos, to which we turn next. 

\medskip
\noindent {\it 5~~~Neutrino Physics}

\smallskip
\noindent
Neutrinos are unique exploratory tools in particle physics. The special nature of their mass makes them sensitive to new physics at very short-distances. Their special propagation properties make them a penetrating probe into the far structure of the Universe and a precious instrument to peek into the dark sectors of the cosmos. Many intriguing open questions in particle physics are linked to the properties of neutrinos.

An active ongoing experimental programme aims at establishing the nature and mass ordering of neutrinos and 
at measuring, with increasing precision, their overall mass scale and mixing parameters (see Chapter~\ref{chap:neut}). An elegant explanation for the lightness of neutrinos in terms of IR Simplicity and separation of scales is encoded in a dimension-five operator with a new-physics scale $\Lambda_\nu$ in the range of $10^{15}$~GeV. This result gives a conclusive proof for the existence of physics beyond the SM. The scale $\Lambda_\nu$ has the dimension of mass divided by coupling squared, so its value could be explained by either a large mass or a small coupling (or a combination of the two). Since neutrino masses require the breaking of both chiral symmetry and lepton number, they are sensitive to fundamental ingredients of the symmetries of the particle world. Furthermore, the neutrino mixing angles show a pattern distinctively different than that observed in the quark sector, exposing another puzzling aspect of the flavour problem. Understanding this structure is a central question in particle physics today. Firmly establishing CP violation in the lepton sector would be a milestone in neutrino physics. A possible consequence of CP violation and neutrino physics is leptogenesis, which is the simplest and most robust known mechanism for generating the cosmic baryon asymmetry in the early Universe. 
Another interesting aspect of the neutrino experimental programme is the search for new light particles that could hide behind the origin of neutrino masses, contributing to the great physics potential offered by the exploration of neutrino physics.

\medskip
\noindent {\it 6~~~Strong CP}

\smallskip
\noindent
There is only one parameter, among those that specify the SM renormalisable interactions, which has not yet been measured: the strong $\theta$ angle. This parameter characterises the vacuum structure of the theory and contributes to the neutron EDM. Current experiments set an upper bound 
% of its physically meaningful definition 
$|\theta |< 10^{-9}$ and no accidental symmetry in the SM can justify such small value. Finding an experimental confirmation of the reason for the surprising smallness of $\theta$ is still an open question.

Out of the several possible proposed solutions, the axion remains the all-time favourite by theory. Based on a spontaneously-broken global symmetry, the axion solution turns $\theta$ into a dynamical variable, which relaxes to zero in the presence of a potential generated by non-perturbative QCD effects. The search for the axion is a central task in particle physics, offering a window into new physics which may take place at very high energies (see Chapter~\ref{chap:dm}). The axion can also have important effects in stellar evolution, in early-Universe dynamics, and its coherent oscillations could explain the dark matter we observe today.

\medskip
\noindent {\it 7~~~Gravity}

\smallskip
\noindent
Gravity is the most familiar of all forces in nature and yet it hides some of the most perplexing open questions in particle physics today. At the classical level, it is elegantly understood as a gauge theory in which the gauge symmetry acts on spacetime coordinates, according to General Relativity. At the quantum level, the theory ``predicts its own destruction" somewhere below the Planck mass, at the extraordinary energies of $10^{19}$~GeV. In spite of the great developments in string theory, the ultimate theory bringing together quantum gravity and the SM has not been identified yet.

Early cosmology and black-hole physics provide the two known training grounds where ideas about gravity in the quantum regime can be tested. The thermodynamical properties of black holes and the information paradox have stimulated new ideas that are revolutionising the approach towards the quantum properties of gravity. Although this research is revealing surprising connections that range from quantum information to condensed-matter physics, this is still a highly speculative and theoretical activity. However, the observation of black hole collisions through gravitational waves has opened a new field that holds promise for experimental tests of modifications of gravity (see Chapter~\ref{chap:cosm}).

Another open question related to gravity is the value of the cosmological constant. From the particle-physics point of view, the amount of dark energy measured by astronomers is ridiculously small. The vacuum energy typically predicted by particle theories contributes to the cosmological constant by some 120 orders of magnitude more than what is observed. The problem is particularly interesting from a particle-physics perspective because it is conceptually identical to the naturalness problem encountered with the Higgs mass, suggesting that there could be hidden connections between the two puzzles.

\medskip
\noindent {\it 8~~~Dark Matter}

\smallskip
\noindent
There is overwhelming observational evidence for the existence of Dark Matter (DM), whose contribution to the mass density of the Universe is 5.3 times larger than for ordinary baryonic matter. While its gravitational imprint is well established at galactic and cosmic scales, the microscopic nature of DM is still a mysterious and outstanding open question. If in corpuscular form, the DM constituents could have masses that vary by some 90 orders of magnitude, ranging from Fuzzy DM of $10^{-22}$~eV to primordial black holes of tens of solar masses. During the last decades, a significant experimental effort has focused on DM masses around 100~GeV:  these are motivated by the  observation that particles in this mass range predicted by supersymmetry or other weak-scale theories automatically lead to a particle density in excellent agreement with the observed DM density. This is usually referred to as the `WIMP miracle'. Recently, there has been growing interest in widening the scope of these searches. On one side, the lack of discoveries of weak-scale particles that could act as mediators in primordial annihilation processes has changed the emphasis, since the choice of masses around 100~GeV was completely driven by specific model considerations, especially related to supersymmetry. A more generic WIMP, which annihilates into gauge bosons via ordinary weak interactions, prefers larger masses. The WIMP miracle occurs when the mass is 1.1~TeV for a weak doublet, 2.9~TeV for a triplet, and even larger masses for larger SU(2) representations. On the other side, renewed interest has flared for DM masses well below the weak scale. These cases are theoretically motivated by axions or axion-like particles, asymmetric DM, light mediators, or non-thermal relics. Experimentally, the search for light DM has stimulated new remarkable ideas using unconventional techniques (see Chapter~\ref{chap:dm}).

Besides the exciting prospect of discovering a new form of matter so common in the Universe, the search for DM is fascinating  because it brings together different fields (particle physics, cosmology, astrophysics) and different experimental techniques (accelerators, underground detection, cosmic rays). Future accelerator-based projects (from high-energy colliders to fixed-target and beam-dump experiments) can contribute to the search for DM in a distinctive and unique way (see Chapter~\ref{chap:bsm}).

\medskip
\noindent {\it 9~~~The Cosmos}

\smallskip
\noindent
One of the greatest successes of particle physics was to show that knowledge derived from very short distances is crucial to understand our Universe at large scales. This path of knowledge started from nuclear physics explaining why stars shine,  and led to the present understanding of galaxy distribution in terms of quantum fluctuations of a primordial field active during inflation. The connection between the Universe at the smallest and largest scales is a monumental conceptual achievement, which has not only produced some of the most mind-boggling results in physics, but also revealed new fundamental open questions. 

One question that future colliders will be able to address is the nature of the electroweak phase transition in the early Universe. While the phase transition is a high-temperature phenomenon that cannot be recreated experimentally, precision measurements of Higgs properties---in particular of the triple-Higgs self-coupling---will give us decisive elements to reconstruct the dynamics that occurred when the Universe changed its vacuum state. According to the SM, the Higgs mechanism took place as a smooth crossover when the Universe cooled down to temperatures below 160~GeV, but the transition could be very different in the presence of new physics. A particularly interesting possibility is that the Universe underwent a first-order phase transition, which would open the door to the exciting prospect of explaining the cosmic baryon asymmetry with weak-scale physics or of observing gravitational waves produced by the abrupt transition at that epoch. Independently of these speculations, testing the nature of the electroweak phase transition is an important task for future colliders that will considerably expand our knowledge about the early history of the Universe.

Inflation and dark energy are two other crucial ingredients of cosmology that require input from particle physics. They are recurrent themes of theoretical physics studies and targets for projects in observational cosmology. Any progress addressing these two fundamental problems would have revolutionary impact on our understanding of the particle world and on our future research priorities. 

\medskip
\noindent {\it 10~~~Dark Sectors and Feebly Interacting Particles}

\smallskip
\noindent
 In the exploration of the unknown, the high-energy frontier remains the best motivated direction to concentrate research efforts in particle physics. Nevertheless, the puzzling questions that confront us require a broad approach, both in terms of experimental strategies and theoretical hypotheses. One alternative research direction that has recently gained momentum is the search for new families of particles which are either very light, but only rarely produced in collisions among ordinary particles, or have very long lifetimes, thus travelling macroscopic distances. These particles are usually referred to as Feebly Interacting Particles (FIP). The hypothetical existence of FIPs is a valid open question in particle physics today.

Superficially FIPs may appear as very unconventional and exotic objects, but we can take the SM for comparison. We know about the existence of very weakly-interacting and penetrating light particles (neutrinos), of particles with relatively long lifetimes due to tiny mass differences (neutrons) or mass hierarchies (weakly-decaying hadrons). Thus, in the presence of hidden sectors with a structure as rich as the SM, it is not unrealistic to expect new particles behaving as FIPs. The existence of FIPs is also motivated by theoretical models for DM, approximate Goldstone bosons, mechanisms for neutrino masses, Higgs naturalness and various hidden sectors. The search for FIPs (see Chapters~\ref{chap:bsm} and \ref{chap:dm}) is an interesting complement to high-energy explorations, with the additional feature of bringing together different experimental strategies ranging from particle physics (collider, beam dump, fixed target, rare decays) to neighbouring fields (DM detection, astrophysics, multi-messenger astronomy, and even atomic or condensed-matter physics).

\subsection*{Advancing particle physics}

Particle physics is a central node of the interconnected network of scientific disciplines that defines human knowledge. It has been able to distill the essence out of natural phenomena and translate it into universal laws expressed in terms of a few mathematical equations. Those laws follow from a handful of fundamental principles and show a remarkable structural unity. Their power lies in allowing us to make certain and precise predictions that have been empirically verified in the domain of simplicity -- the Universe at very small and very large distance scales. 
%and that are likely to form the foundation for the emergent laws governing the domain of complexity.

  The success of particle physics relies on the ability to build experiments where controlled phenomena can be studied, leading to certain and precise measurements. The combination of precise predictions and precise measurements is the hallmark of particle physics among the sciences, which allows us to ask fundamental questions about the Universe and obtain definite answers. 
%The power of precision is also  what allows to formulate the pressing crucial unsolved questions.

 The present success of the Standard Model is the very reason why we can embark on future missions. This success 
 %because it
  gives us a reliable and solid starting point to formulate consistently our questions and to advance cogently and systematically into the exploration of the unknown.   In particular, the pressing  open questions have been formulated precisely and rigorously.  Their crucial nature is tantalising: they may herald a change of paradigm awaiting discovery. 
   Addressing those questions 
%many questions that confront particle physics 
  requires a diversified scientific strategy. Within this broad research programme, high-energy colliders are an indispensable and irreplaceable tool to pursue our exploration of the fundamental laws of nature.
 
\end{document}



