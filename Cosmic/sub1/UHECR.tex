Cosmic accelerators deliver the highest energy protons, photons, and neutrinos for probing new physics. The physics scope of the research field ``Ultra-High Energy Cosmic Rays'' (UHECR) is focused on the detection of 
cosmic rays of energies above PeV and the related physics questions. The detection is performed with 
large-scale experimental infrastructures via the observation of extensive air showers generated by the impinging primary particles. Approximately 98 \% of the flux corresponds to hadronic particles, and less than 1\% to anti-matter, photons or neutrinos.

The main goals of the research are the characterization of the particles reaching the Earth's atmosphere and the unveiling of their sources, the understanding of the acceleration and propagation physics of these particles, 
the search for extreme high energy neutrinos and photons, as well as the study of
particle physics at energies and in phase space regions not accessible at man-made accelerators, i.e.\ interaction physics and cross section measurements with energies typically above 100 TeV. Furthermore, the observatories of UHECR are important instruments for the Multi-Messenger
observations of the Universe due to their capabilities in extending the energy range on photon 
and neutrino diffuse fluxes compared to dedicated Gamma-ray and Neutrino facilities as well the  study of possible ultra-high energy neutrino and photon signals in correlation with gravitational wave events. In addition, this allows  also an extension of the mass range in indirect searches of Dark Matter. 

The interpretation of air shower data heavily relies on our knowledge of particle interaction, production, and decay over a very wide range of energies and phase space regions. Particle physics theory and measurements made at accelerators provide indispensable and complementary input for understanding extensive air showers. 
Here the role of fixed-target and collider experiments for providing important data is recognized and a coherent  measurement program  included in the plans for accelerator-based experiments would significantly improve the accuracy of the UHECR measurements. In particular, the research field supports plans of an LHC
run with light ions, like proton-oxygen, which would fill a very important gap in data needed
for air shower physics. Similarly, fixed-target and collider measurements with very good forward coverage for hadron production will be very valuable. Such an experimental program and corresponding  joint theoretical studies of questions in the overlap region between particle and astro-particle physics is of fundamental importance for making scientific progress. One example is the development and tuning of hadronic event generators needed in high-energy physics and air shower simulations. To make just one example, both theoretical and experimental work is needed to address the well-established muon discrepancy in air showers. While the measured muon excess relative to predictions probably is related to shortcomings in the simulated hadronic interactions it could also indicate new particle physics at energies beyond the reach of LHC.

At the same time, studying UHECR generated air showers provides a window to energies far beyond 
those accessible at existing accelerators and also emphasizes phase space regions of particle production
that can not be covered in collider experiments. Important are here the measurements of proton-air cross 
sections above 20 TeV c.m.~energy. 
Moreover, cosmic ray generated air showers provide a very good possibility for searching physics Beyond the 
Standard Model for various classes of models,  including production of micro-black holes and other heavy states.
The combination of UHECR data with astrophysical information, in addition, gives  physics reach  to 
fundamental phenomena, like providing important constraints on theories of quantum gravity involving 
Lorentz invariance violation.
%\vspace*{0.5cm}
%Probably off-topic here, but need to be mentioned: 
%Connections between particle physics and astroparticle physics (including UHECR) are also tight in technology and society related topics: \\
%Technology:  detector developments, readout electronics, (Monte-Carlo) software, handling large infrastructures, computing models, Big Data Analytics, ... \\
%Society: ‚FAIR data life cycle, outreach, education / training, ...
