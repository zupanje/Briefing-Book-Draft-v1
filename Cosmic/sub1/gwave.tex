%\paragraph{Gravitational Waves: A new window on the Universe} 
\section{Gravitational waves}
The past four years have seen a new revolution in astronomy with the Nobel prize-winning detection of gravitational waves from binary black holes and neutron stars by the US LIGO and the European Virgo interferometers. These discoveries have begun to unravel some of the most fundamental questions in physics and astronomy. They provide new tests of Einstein's general relativity, offer black holes as a possible component of the elusive dark matter,  facilitate determination of the equation of state of dense nuclear matter in neutron stars, allow a completely independent measurement of the Hubble parameter. They also confirmed that gravitational waves travel at (about) the same speed as light thereby ruling out many modified gravity theories that were invoked to account for dark energy.  

The next generation of gravitational-wave detectors such as the Einstein Telescope in Europe and Cosmic Explorer in US will enable unprecedented and unique science in extreme gravity, fundamental physics and cosmology. They will provide answers to many puzzling questions related to the nature of dark matter, the nature of gravity, the nature of compact objects, cosmology, and matter in extreme environment like the following.
Is dark matter composed of particles, dark objects or modifications of gravitational interactions?
Is Einstein's general relativity the ultimate theory of spacetime?  What building-block principles and symmetries in nature invoked in the description of gravity can be challenged?
Are black holes and neutron stars the only astrophysical extreme compact objects in the Universe? Are there subtle signatures of quantum gravity in the spacetime geometry of these compact objects?
What phase transitions took place in the early history of the Universe and what are their energy scales? How do cosmological parameters vary with redshift?
What is the equation of state of the densest matter in the universe as may be found in neutron star cores? Is there unknown physics in the state of ultra-high density matter, e.g. phase transition from nucleons to quark-gluon plasma?

%{\em The nature of dark matter:} {Is dark matter composed of particles, dark objects or modifications of gravitational interactions?} {\em The nature of gravity:} {Is Einstein's general relativity the ultimate theory of spacetime?  What building-block principles and symmetries in nature invoked in the description of gravity can be challenged?} {\em The nature of compact objects:} {Are black holes and neutron stars the only astrophysical extreme compact objects in the Universe? Are there subtle signatures of quantum gravity in the spacetime geometry of these compact objects?} {\em Cosmology and the early universe:} {What phase transitions took place in the early history of the Universe and what are their energy scales? How do cosmological parameters vary with redshift?} {\em Matter in extreme environs:} {What is the equation of state of the densest matter in the universe as may be found in neutron star cores? Is there unknown physics in the state of ultra-high density matter, e.g. phase transition from nucleons to quark-gluon plasma?}

%\paragraph{Einstein Telescope:} 
The Einstein gravitational-wave Telescope (ET) will be an observatory of the third generation, aiming to reach a sensitivity for signals emitted by astrophysical and cosmological sources about a factor of ten better than the design sensitivity of the advanced detectors currently in operation. To reduce the effects of the residual seismic motion, ET will be located underground at a depth of about 100 m to 200 m and, in the complete configuration, it will consist of three nested detectors, each in turn composed of two interferometers. The topology of each interferometer will be the dual-recycled Michelson layout with Fabry-Perot arm cavities, with an arm-length of 10 km. The configuration of each detector devotes one interferometer to the detection of the low-frequency (LF) components of the signal (2--40 Hz) while the other one is dedicated to the high-frequency (HF) components. In the former (ET-LF), operating at cryogenic temperature, the thermal, seismic, gravity gradient and radiation pressure noise sources will be particularly suppressed; in the latter (ET-HF) the sensitivity at high frequencies will be improved by high laser light power circulating in the Fabry-Perot cavities, and by the use of frequency-dependent squeezed light technologies.

ET science and technology greatly reflects CERN's own pursuit of fundamental physics and precision engineering that was required to build and operate the LHC. The core areas where CERN's expertise can benefit ET are underground infrastructure, cryogenics, vacuum, material and surface science, electronics and data acquisition, and computing.
%\vfill
\vspace*{-4mm}