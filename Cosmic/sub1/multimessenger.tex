\section{Multi-messenger astroparticle physics}
The spectacular observations of Supernova SN1987A in both MeV neutrinos as well as electromagnetic radiation provided an early taste of the enormous scientific potential of multimessenger astronomy, not just for astronomy but also for fundamental physics. It then took 30 years to conquer the extra-galactic distance scale with the result that  today we can observe events such as neutron star merger in both Gravitational Waves and electromagnetic radiation, or coincident high-energy cosmic neutrinos from active galaxies that are flaring in high-energy gamma rays. New generations of observatories with significantly improved sensitivity are either in construction or being planned,  further accelerating the discovery rates and exploiting the scientific opportunities provided by the combination of the messengers. 

Gamma-ray observatories, as well as neutrino or cosmic ray detectors such as AMS II (a recognized experiment at CERN), allow for sensitive searches of dark matter via  indirect signatures. In fact, the European Cherenkov Telescope Array (CTA), which just entered the construction phase, will have sensitivity to probe the natural WIMP annihilation cross-sections in the TeV mass range. 

%Cosmic neutrinos have been observed with energies reaching $10^{16}$~eV.  %Their cross-section can be probed indirectly through their absorption in the Earth. The flavour composition of cosmic neutrinos, predicted to lie in a narrow range for various source scenarios and standard neutrino oscillations, probes BSM physics and the cosmic fabric itself through propagation effects over cosmic baselines. 

The observations of high-energy neutrinos  in temporal coincidence with the flaring Blazar \cite{IceCube:2018dnn} has not
only launched a new era in the study of the origins of high-energy
cosmic rays, but also made possible a breakthrough in the exploration
of Lorentz symmetry using neutrinos. A new generation of neutrino detectors, such as GVD, IceCube Gen2, KM3NeT or GRAND aim at increasing the available event statistics, as well as the energy range, and  will significantly improve the sensitivity to new physics.
\vspace*{-2mm}
 
%The energy frontier is established through cosmic rays, protons and heavier nuclei, that reach energies beyond $10^{20}$~eV.  Cosmic magnetic fields scramble their directions,  making the identification of their sources a challenging task (thus motivating neutrino astronomy). However, already now the events that develop as huge air showers in the atmosphere are used to study interaction cross-sections, with significant impact for hadronic interaction models. 

%Last but not least, gravitational waves---the newest messenger on the block---provides for a very rich physics harvest, ranging from tests of extreme gravity, axions and dark matter, to nuclear physics. With the Einstein Telescope, a ground based interferometer of the 3rd generation is in the planning in Europe. It  requires the cooperation of leading institutions in particle physics and astronomy, along with their funding agencies, to make it a success. Interestingly, much of the required expertise appears to be available in accelerator laboratories already. 

