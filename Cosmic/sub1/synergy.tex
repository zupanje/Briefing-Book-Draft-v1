\section{Synergies with HEP}
The field of astroparticle physics has deep connections at many levels with the field of particle physics. The case of neutrinos is particularly revealing. The cosmos provides bright sources of neutrinos with the Sun, cosmic rays producing atmospheric neutrinos, UHE neutrinos etc. The study of these sources has revealed neutrino oscillations, a phenomenon that proves the existence of physics BSM. The precision measurements of the neutrino properties still relies on solar and atmospheric neutrinos for the determination of several mass and mixing parameters. This deep complementarity might continue in the  future for determination of the neutrino mass ordering, the study of sterile neutrinos and maybe other surprises with UHE neutrinos.

The case of the searches of dark matter (see Chapter\ref{chap:dm}) is also a very important sector for these synergies, with very sensitive indirect searches in a 
variety of modes being performed by large astroparticle detectors like HESS, Antares or IceCube, and especially with the CTA observatory starting operation in 2022.

Large underground neutrino detectors are used in long baseline accelerator experiments and in astroparticle physics with strong synergies between the two domains. For instance, the multi-PMT modules developed for KM3Net are considered for the Hyper-Kamiokande project. The same applies at the level of the physics results, as all these detectors will be sensitive to similar sources, like supernova explosions, or atmospheric neutrinos, and make complementary measurements of the PMNS matrix. The determination of the neutrino mass ordering is particularly significant, as both accelerator-based experiments (NOVA, DUNE, Hyper-Kamiokande) and astroparticle experiments (Super-Kamiokande, Icecube, KM3Net) can contribute. 
IceCube has already started the study of NSI and sterile neutrinos with UHE neutrinos and has a unique sensitivity to tau neutrinos.
The same applies for cosmic ray physics, but also gravitational wave observatories. Some of the latter, like the ET, use infrastructure and techniques that are very similar to those deployed for large underground accelerator complex.

There are therefore multiple synergies between particle and astroparticle physics at the level of infrastructure (large underground excavations, vacuum, engineering and management of large projects), detector (including photosensor and other detection techniques, electronics, computing), interaction models (for neutrino and cosmic rays for instance), physics, etc. All these synergies might clearly benefit from a closer collaboration between CERN and the astroparticle community, to be developed through focused discussion between CERN and APPEC.