%\documentclass[12pt,twoside,a4paper]{cernrep} 
%\usepackage{rep_common}
%\pagestyle{plain}
% --
% -- The bibliography file paths for the main volume and all parts
% --
%\def\bibfiles{\main/bib/chapter, \main/section1/bib/section,
%\main/section2/bib/section, \main/section3/bib/section,\ main/section4/bib/section}
% --
%\providecommand{\biblio}{\nocite{article-minimal}\bibliographystyle{report}\clearpage\bibliography{\bibfiles}}  % *Modification: added `\main/` to specify relative file location.

%\usepackage{lineno}
%\linenumbers

\begin{document}
%\newcommand{\main}{.}
%\newcommand{\FCChh}{FCC-hh\xspace}
%\newcommand{\FCCee}{FCC-ee\xspace}
%\newcommand{\FCChhLowE}{FCC-hh, low energy\xspace}
%\newcommand{\FCCeh}{FCC-eh\xspace}
%\newcommand{\CLIC}{CLIC\xspace}
%\newcommand{\CEPC}{CEPC\xspace}
%\newcommand{\ILC}{ILC\xspace}
%\newcommand{\CLICThreeHundredEighty}{CLIC\textunderscript{380}\xspace}
%\newcommand{\CLICThreeThousand}{CLIC\textunderscript{3000}\xspace}
%\newcommand{\ILCTwoHundredFifty}{ILC\textunderscript{350}\xspace}
%\newcommand{\ILCFiveHundred}{ILC\textunderscript{500}\xspace}
%\newcommand{\HLLHC}{HL-LHC\xspace}
%\newcommand{\HELHC}{HE-LHC\xspace}

%\def\biblio{}
\linenumbers
% this is the extra information to be used for the general sections.
\chapter{Dark Matter and Dark Sector}
\label{chap:dm}
%\title{
%  Dark matter and Dark sector \\ 
%  {\small for the Update of the European Strategy for Particle Physics 2019}
%  }
%\author{\rm 
\vspace*{-10mm}
\centerline{Input for the Update of the European Strategy for Particle Physics 2020} \\
\vspace*{9mm}
%{\it The Open Symposium in Granada (13--16 May 2019) was organized around 8 themes that are reflected in the corresponding chapters of this document, compiled by the following authors} \\
\begin{center}
{\vspace*{-3mm}\bf Electroweak Physics:}\quad \rm Richard Keith Ellis$^1$, Beate Heinemann$^{2,3}$ {\it (Conveners)} \\
Jorge de Blas$^{4,5}$, Maria Cepeda$^6$, Christophe Grojean$^{2,7}$, Fabio Maltoni$^{8,9}$, Aleandro Nisati$^{10}$, Elisabeth Petit$^{11}$, Riccardo Rattazzi$^{12}$, Wouter Verkerke$^{13}$ {\it (Contributors)}\\
{\vspace*{3mm}\bf Strong Interactions:}\quad \rm Jorgen D'Hondt$^7$, Krzysztof Redlich$^8$ {\it (Conveners)}\\
Anton Andronic, Ferenc Sikl\'{e}r {\it (Scientific Secretaries)} \\
Nestor~Armesto$^5$, Dani\"el~Boer$^6$, David~d'Enterria$^7$, Tetyana Galatyuk,  Thomas~Gehrmann $^8$, Klaus~Kirch$^9$, Uta~Klein$^{10}$, Jean-Philippe~Lansberg$^{11}$, Gavin P. Salam$^{12}$, Gunar Schnell$^{13}$, Johanna Stachel$^{14}$,  Tanguy~Pierog$^{15}$, Hartmut Wittig$^{16}$,   Urs~Wiedemann$^{17}${\it (Contributors)} \\
{\vspace*{3mm}\bf Flavour Physics:}\quad 
Belen Gavela, Antonio Zoccoli {\it (Conveners)} \\
Sandra Malvezzi, Ana Teixeira, Jure Zupan {\it (Scientific Secretaries)} \\
{\vspace*{3mm}\bf Neutrino Physics \& Cosmic Messengers:}\quad
Stan Bentvelsen, Marco Zito {\it (Conveners)} \\
Albert De Roeck $^4$, Bonnie Fleming $^5$, Francis Halzen $^6$, Andreas Haungs $^7$, Marek Kowalski$^8$, Susanne Mertens$^9$, Mauro Mezzetto$^{10}$, Silvia Pascoli$^{11}$, Bangalore~Sathyaprakash $^{12}$, Nicola~Serra $^{13}$, Thomas Schwetz$^7$ {\it (Contributors)} \\
{\vspace*{3mm}\bf Beyond the Standard Model:}\quad
Gian F.~Giudice, Paris Sphicas {\it (Conveners)} \\
Juan Alcaraz Maestre, Caterina Doglioni, Gaia Lanfranchi, Monica D'Onofrio, Matthew~McCullough, Gilad Perez, Philipp Roloff, Veronica Sanz, Andreas Weiler, Andrea~Wulzer~{\it(Contributors)} \\
{\vspace*{3mm}\bf Dark Matter and Dark Sector:}\quad
Shoji Asai, Marcela Carena {\it (Conveners)} \\
Babette D\"{o}brich, Caterina Doglioni, Joerg Jaeckel, Gordan Krnjaic, Jocelyn Monroe, Konstantinos Petridis, Christoph Weniger {\it (Scientific Secretaries)} \\
{\vspace*{3mm}\bf Accelerator Science and Technology:}\quad
Caterina Biscari, Leonid Rivkin {\it (Conveners)} \\
Philip Burrows, Frank Zimmermann {\it (Scientific Secretaries)}  \\
Michael Benedikt$^4$, Edda Gschwendtner$^4$, Erk Jensen$^{4}$, Mike Lamont$^{4}$,  Wim~Leemans$^7$,  Lucio~Rossi$^4$, Daniel Schulte$^4$, Mike Seidel$^2$, Vladimir Shiltsev$^6$, 
Steinar~Stapnes$^4$,  Akira~Yamamoto$^{4,5}$ {\it (Contributors)} \\
{\vspace*{3mm}\bf Instrumentation and Computing:}\quad
Xinchou Lou, Brigitte Vachon {\it (Conveners)} \\
Roger Jones, Emilia Leogrande {\it (Scientific Secretaries)} \\
Ian Bird$^4$, Amber Boehnlein$^5$, Simone Campana$^4$, Ariella Cattai$^4$, Didier~Contardo$^6$, Cinzia~Da~Via$^7$, Francesco Forti$^8$,  Maria Girone$^4$, Matthias~Kasemann$^9$, Weidon Li$^1$, Lucie~Linssen$^4$, Felix Sefkow$^9$, Graeme Stewart$^4$ {\it (Contributors)} \\
\vspace*{5mm}
{\bf Editors:\quad}Halina Abramowicz, Roger Forty, and the Conveners
\end{center}}

\institute{\footnotesize\vspace*{6mm}
$^1$ IPPP, University of Durham, Durham DH13LE, UK \\
$^2$ Deutsches Elektronen-Synchrotron (DESY), Hamburg, 22607, Germany \\
$^3$ Albert-Ludwigs-Universit\:{a}t Freiburg, Freiburg, 79104, Germany \\
$^4$ Centro de Investigaciones Energ\'{e}ticas, Medioambientales y Tecnol\'{o}gicas (CIEMAT), Avda.\ Complutense 40, 28040, Madrid, Spain \\
$^5$ Centre for Cosmology, Particle Physics and Phenomenology, Universit\'{e} catholique de Louvain, Louvain-la-Neuve, 1348, Belgium \\
$^6$ Dipartimento di Fisica e Astronomia, Universit\`{a}a di Bologna and INFN, Sezione di Bologna, via Irnerio 46, 40126 Bologna, Italy \\
{\it etc.\ for the other affiliations (to be completed for the next draft)}}




%\begin{abstract}
%The compelling evidence for dark matter from galactic and cosmological observations, together with the possible existence of associated dark sectors, provides a promising path in the search for physics beyond the Standard Model. The mass of dark matter particles  could be anything from as light as  $10^{-22} {\rm eV}$ to as heavy as  tens of solar masses.
%We focus on two leading paradigms, thermal dark matter %(including WIMPs) and non-thermal, ultralight dark matter. %(including QCD-axions). Thermal dark matter motivates more specific mass scales in the KeV/MeV to 100 TeV mass scale, with the low mass region demanding light mediator/s that in themselves are a search target.  Different search techniques are applicable depending on the mass range, including accelerator based searches at colliders and fixed target experiments, and multiple direct detection experiments, some involving low threshold techniques. These techniques are targeted at WIMPS and  hidden sector dark matter- including  sub-GeV dark matter - as well as the related mediator particles. The searches for the well motivated QCD-axion and other ultralight axion-like/weakly interacting particles aims at the detection of particles  produced in the lab and in the sun or at the direct detection of  halo dark matter. The understanding of the dark  matter properties will demand signals from multiple experiments, together with compatibility with indirect measurements from dark matter annihilation products.
%This chapter concentrates on the proposals for dark matter, its mediators and more general dark sector exploration submitted to the ESPP update process. Complementary information appears in sections of other chapters.

%%%%%%%%%%%%%%%%%%%%%%%%%%%%%%%%%
%
%Three different approaches using accelerators and these technologies
%cover completely the wide mass range of dark matter and dark sector. 
%Colliders cover the high mass region, so-called WIMP, and
%\FCChh covers almost all the parameter space of the SUperSYmmetric 
%dark matter candidates,
%which are thermally produced in our Universe.
%Middle mass rage between ${\rm  MeV}$ and ${\rm GeV}$ is covered by 
%the beam dump and fixed target experiments, like SHiP.
%Ultra-light dark matter, such as Axion, is also good candedate of dark matter.
%Searches for Such a light particle are performed by the various experiments, in which
%accelerator techniques, for examples Magnet, Vacuum, cryogenics and fine optics,
%are key technologies.
%With the collaboration with Astronomical approaches,
%direct and indirect detection of dark matter,
%properties of the dark matter and the whole picture of theory beyond the SM can be determined.

%\end{abstract}


%\maketitle

% -- Set the level of the TOC: 2 means including subsection, 1 means
% including section 
%\setcounter{tocdepth}{2}
%\tableofcontents
%\maketitle
~\\
~\\
~\\
%\newpage


\noindent This chapter is based on material submitted by the particle physics community to the  ESPP update process that is relevant for dark matter and dark sectors exploration. Section 1 provides an introduction to the topic. Section 2 briefly highlights current results and potential of dark matter astrophyiscal probes.
%briefly reports on the status and prospects for Direct and Indirect dark matter Detection searches (DD and ID). 
Dark Matter (DM)  and Dark Sector (DS) searches at colliders are discussed in the BSM chapter of this briefing book and are briefly touched upon in Sect.~3 of this chapter through their complementarity with dark matter direct and indirect  detection experiments. Section 4 concentrates on accelerator based DM/DS searches at fixed target and beam dump experiments. Section 5 discusses axion and Axion-Like Particle (ALP) searches. Section 6 concludes on the main findings of the preparatory group exercise for this chapter. More detailed information can be found in %~\ref{note:Support_Note_Complementarity}.
{\bf (reference to support note)}.


\section{Introduction}

%{\bf test for reference~\cite{Asai:2007sw}, first proposal to detect disappearing track.}

There is compelling evidence from galactic and cosmological observations that DM exists, and detecting DM in the laboratory is one of the greatest challenges of particle physics \cite{bertone:2004pz}. Since DM is the dominant form of matter in the universe, it is highly plausible that there is also a richer 
Hidden Sector (HS), 
whose constituents could include multiple species of massive particles, one or more which might mix with Standard Model (SM) particles such as the Higgs boson, the photon or neutrinos, via so called HS-SM portal operators (see \cite{Battaglieri:2017aum} for
a review).
%The HS may also contain new force carriers, such as dark photons, that mediate interactions within the HS or between the HS and SM particles.
A hidden sector that contains dark matter is more generically called a Dark Sector (DS), and includes at least a  mediator connecting  HS particles and SM particles. The SM operator interacting/mixing with the mediator is often referred to as a portal (e.g Higgs portal).
%%%%
%SM portal to DM in the DS or, if the mediator is called the portal itself, will be a DS portal to the SM.
%%%%%%
%The mediator-SM particle interaction is often referred to as a portal (e.g Higgs portal).
%and the mediator connecting  HS particles and SM particles is often called a "portal". %% 
In the most simplified  DS realization, with just  DM as its only component, a new, beyond the standard model (BSM) particle may act as the mediator.
Such mediator can also be the force carrier of a new gauge group under which the
SM particles are charged, e.g. U(1)$_{B-L}$ \cite{Ilten:2018crw,Bauer:2018onh}.


%(e.g (axial)scalar or (axial) vector mediators). 

%%%%%%%%%%%%
%Dark Matter could also be explained as part of an extended BSM sector with many new states carrying SM quantum numbers, and a preserved new symmetry distinguishing between the SM and BSM particles that renders the DM particles stable. The latter approach, inspired in BSM theories build to solve additional mysteries of particle physicis has been the most explored scenario in the past decades (e.g, Supersymmetry with WIMP DM)
%%%%%%%%

%Such scenarios are more generically known as Hidden Sectors (HS),
%and the mediator interactions between the HS particles and SM particles is called a "portal".
% Dark sector is a hidden sector with DM that talks to us through a portal.

While the observational evidence for dark matter is exceptionally convincing, our current level of ignorance of the basic properties of dark matter is remarkable 
%exceptionally high 
\cite{Bertone:2016nfn}. The mass of dark matter particles could be anything from as light as $10^{-22}$ eV \cite{Hu:2000ke} to as heavy as primordial black holes of tens of solar masses \cite{Bird:2016dcv}. The lower mass limit comes from the requirement that dark matter particles can have a sufficiently short de Broglie wavelength to form dwarf galaxies and galactic sub-halos, while the upper limit is set by observational limits on massive compact halo objects (MACHOs) and CMB anisotropies \cite{Brandt:2016aco,Nakama:2017xvq,Ali-Haimoud:2016mbv}. Within this very large mass range it is useful to define some distinct families of possibilities. 

Considering the DM production mechanism in the early universe provides a useful classification. We shall concentrate  here on  two prominent examples. %Assuming thermal equilibrium in the early Universe 
Assuming the DM is produced thermally, through interactions with the SM in the early universe, 
narrows the viable DM masses significantly, to the few keV to 100 TeV range (e.g.\ WIMPS and hidden sector particles). 
In the case of standard cosmology, theoretical models become highly predictive, however, in nonstandard cosmology a wide range of models can produce the observed dark 
matter abundance.
%In this range, for DM in equilibrium with SM particles, theoretical models become highly predictive as they do not depend on assumptions about unknown cosmological epochs such as inflation or baryogenesis. 
Alternatively, ultralight particles in the sub-eV must be produced non-thermally (e.g.\ QCD axion and axion-like particles) 
%Prominent examples are the QCD axion and axion-like particles that we discuss below.

We are equally ignorant of dark matter interactions. As part of a dark sector, 
dark matter may interact only gravitationally with SM matter or it may interact via DS mediators (neutral under the SM gauge group) interacting/mixing through a SM portal, as well as through new particles charged under the SM gauge group that couple to the dark matter directly.
%if the DM mass is above the GeV scale~\cite{Lee:1977ua}.
Alternatively, dark matter may carry SM charges itself as  part of an extended BSM sector with many new states carrying SM quantum numbers. In such case  a preserved  symmetry distinguishing between the SM and BSM particles  renders the DM particle stable (e.g.\ as in supersymmetry). 
In this latter approach, inspired by BSM theories built to solve additional mysteries of particle physics, we refer to the dark matter as Weakly Interacting Massive Particles (WIMPs).
%In such case the DM may interact at tree level, for example via the Higgs boson, or through loop-suppressed exchange of heavy SM particles such as the W boson.
%, as well as via  similar  processes involving weakly coupled BSM particles.

The history of dark matter direct detection  experiments has been dominated by WIMP searches, motivated by the 
%both by the WIMP candidates provided by models of supersymmetry, and by the
 so-called ``WIMP miracle'': the qualitative observation that particles with roughly weak scale $\cal{O}$(100) GeV masses, and weak scale interactions with SM particles, will end up with roughly the observed thermal relic density after freeze out in standard big bang cosmology (see \cite{Jungman:1995df} for a review). WIMP scenarios, of $\cal{O}$(100) GeV mass dark matter particles interacting with detectors via $Z$ boson exchange, has already been strongly constrained by the impressive current limits of multiple overlapping direct WIMP searches. Current and proposed WIMP direct detection experiments will push the sensitivity %to more general WIMPs 
 down to the so-called ``neutrino floor'', where background from known astrophysical neutrino sources will swamp the expected signal of nuclear recoils from interactions with WIMPs from the galactic halo \cite{Billard:2013qya}. New strategies are being developed to conquer the neutrino floor and enable exploration beyond it.
 Collider searches for  WIMP candidates, such as those provided by models of supersymmetry,  are a main focus of interest at the HL-LHC and provide a strong case for a future high energy hadron collider.

WIMP dark matter is not the only example of plausible dark matter particles that can be explained as thermal relics from early universe cosmology. 
Hidden sector dark matter can provide such thermal relics in the mass range 
between about a few keV \cite{Safarzadeh:2018hhg}  and  about 100~TeV, with unitarity constraints yielding the upper bound \cite{griest:1989wd}. In principle, such relic dark matter could interact with laboratory detectors via new exotic, feeble interactions.
Laboratory detection of keV-GeV thermal dark matter, often referred to as light dark matter (LDM), requires deploying novel detector technologies and/or new detection strategies, since existing strategies typically fail below the GeV scale. 
Interestingly enough, the phenomenology of low mass region (keV--GeV) thermal dark matter is quite different from the standard WIMP. In particular, in order to 
 deplete the DM density at freeze out to agree with the currently observed values, 
it typically demands the existence of light mediator/s 
%\cite{Lee-Weinberg}, 
that also give rise to novel laboratory searches independently of their connection to DM. 
%that in themselves are a search target.
New accelerator-based searches, in which a high intensity, relativistic beam of particles (electrons, protons, muons) impact on a dense target, offer a   compelling path to probe  light dark matter and dark sector portals. This accelerator-based fixed target experiments can exploit beam dump and missing energy/momentum techniques and will have unprecedented sensitivity to dark sector realizations at the CERN SPS and/or the SLAC LCLS-II facilities.  
They are complementary to electron recoil signals at dedicated low-mass direct detection experiments. 

%In the case of ultralight, non thermal DM with mass less that about an eV, the measured dark matter relic density implies that the occupation numbers of these ultralight dark states are very high. Therefore such ultralight DM (lighter that about 0.7 keV) must be bosonic, and the effects of halo dark matter in terrestrial experiments are best described as wavelike disturbances. 
 In the case of ultralight, non thermal DM with mass less that about an eV, the measured galactic dark matter density implies that the occupation numbers of these ultralight dark states are very high. Therefore such ultralight DM (lighter that about 0.7 keV) must be bosonic \cite{Tremaine:1979we}, and 
 the effects of halo dark matter in terrestrial experiments are best described as wavelike disturbances. 
 A particularly compelling candidate for ultralight dark matter is the QCD axion, whose mass can be in the range 
$10^{-12}$ to ${\rm  10^{-3}eV}$ (see \cite{Marsh:2015xka} for a review). 
%Thus, "wavelike" dark matter corresponds to dark matter particles with mass less than about an eV; in this case the measured dark matter relic density implies that the occupation numbers of these ultralight dark states are very high. This implies that ultralight dark matter must be bosonic (if lighter than about 0.7 keV), and that the effects of halo dark matter in terrestrial experiments are best described as wavelike disturbances. 
If QCD axions are a major component of dark matter, they could  be detected as coherent waves from the halo. Quite generally, axions can also be produced as higher energy particles by our Sun and detected by terrestrial experiments. Many beyond the SM scenarios contain axion-like particles and other sub-eV dark sector particles that might eventually also be produced at the laboratory 
without relying on cosmological and astrophysical assumptions. Ultralight DM searches motivate a broad variety of non-accelerator experiments, which however profit crucially from technologies developed in accelerator-based research \cite{Battaglieri:2017aum}.

%but with lower fluxes.

%In principle such relic dark matter could disturb laboratory detectors via new exotic "feeble" interactions. Of particular recent interest is the dark matter mass range between about an MeV and a GeV, where the nuclear recoil signal for direct detection is accessible in dedicated low-mass WIMP experiments.
%
% JRM-- the old sentence said this: "where the nuclear recoil signal for direct detection is too feeble for most current or planned WIMP experiments." There are dedicated technologies (e.g. CRESST, SENSEI, DAMIC) that are direct detection searches that are optmized for this mass range.  So I've reworded this sentence to reflect this.
%

%Big questions addressed by the dark matter working group: 

%\begin{enumerate}
%\item	How do we search for DM, depending on its properties?
%What are the main differences between light Hidden Sector dark matter and WIMPs?    
%How broad is the parameter space for the QCD axion? 

%\item	What are the most promising experimental programs, approved or proposed, to probe the different dark matter possibilities in a compelling manner?       

%\item How to compare results of different experiments in a more model-independent way? 

%\item How will direct and indirect dark matter Detection experiments inform/guide accelerator searches and vice-versa? 

%\end{enumerate}



% Introduction including Astroparticle 
\subfile{\main/Darkmatter/section1/section}
%\newpage

%% Direct and Indirect detection
%%\subfile{\main/section2/section}
%%\newpage

% Dark matter @ collider 
\subfile{\main/Darkmatter/section2/section}
%\newpage

% Dark matter @ beam dump 
\subfile{\main/Darkmatter/section3/section}
%\newpage

%ALPS
\subfile{\main/Darkmatter/section4/section}
%\newpage


\section{Conclusions}
%Cosmological observations of Dark Matter and the baryon asymmetry in the Universe provide overwhelming evidence of physics beyond the SM.
%Probing for dark matter and dark sectors is the promising way to go beyond the SM, and it requires numerous experiments and techniques in order to cover the many possibilities.
Gravitational and cosmological observations provide overwhelming evidence of the existence of DM and of its predominance over other types of matter in the Universe. They also provide a compelling  proof of physics beyond the SM. Given our high current level of ignorance about the basic properties of dark matter, a comprehensive suite of  experiments and techniques are required in order to cover the many possibilities.
In this chapter we have discussed experiments that can explore a broad range of dark matter masses, from ultralight dark matter (below the eV) to thermal DM, either light (a few keV to GeV) or heavy/WIMP-like (GeV to 100 TeV).
Many of such experiments can also explore a dark sector with dark mediators and other feeble interacting particles than may be present within DM scenarios.

%In the case of LDM, comparably light mediators are necessary and such mass regions  has been until now quite underexplored.
Accelerator-based, beam dump and fixed target experiments such as SHiP, LDMX, NA62$^{++}$ and NA64$^{++}$ can perform sensitive and comprehensive searches of sub-GeV Dark matter and its associated dark sector mediators. They will broadly test models of thermal LDM that are as yet underexplored. CERN has the opportunity to play a leading role in these searches by fully exploiting the opportunities offered by the SPS and the foreseen Beam Dump Facility.

%WIMP DM has been, instead, extensively explored at colliders and in direct dark matter detection experiments, and still remains a viable candidate to be searched for. 
Prospective future colliders (ILC/CLIC, FCC-ee/hh/eh and HL/HE-LHC) have excellent potential to explore models of thermal DM in the GeV -- 10 TeV mass range, notably including WIMPs as well as models with different types of DM mediators. 
Feebly interacting particles, plausibly produced at colliders can also be searched for at prospective new detectors (e.g.\ FASER, CODEXb, MATHUSLA) further away from the beam interaction points. 
%and can cover most of the mass range available to thermal relics. 
These new search strategies are complementary to accelerator-based, fixed target (beam dumps and missing energy/momentum) experiments as well as standard collider searches depending on the mass region and the coupling strength between the SM and DM/dark sector. 
%To claim a discovery, for example,  in the appealing, extensive region with thermal DM, multiple techniques are necessary, including confirmation by direct detection experiments and compatibility with annihilation products from indirect astrophysical searches. 

The search for ultralight DM particles like the axion has gained significant momentum. IAXO  provides a compelling opportunity to extend the search for axions. In addition, haloscopes such as MadMax could directly detect axion dark matter, whereas ALPS-II and other prospective light-shining-through-wall experiments can provide competitive pure laboratory tests.




%Dark sectors models can provide answers to fundamental questions in particle physics, such as the origin of Dark Matter, the Baryon Asymmetry in the Universe, the origin of neutrino masses. This, coupled with a lack of a clear hint for the energy scale of new physics from the energy frontier, means that searches for Dark sectors and LDM have leading part to play in the European particle physics strategy over the next decade, with CERN playing a leading part through the proposed BDF.
%The search for a Dark sector requires  numerous experiments and techniques in order to cover the large parameter space and plethora of models. Beam dump and fixed target experiments such as SHiP, LDMX and NA64++, will perform the most sensitive and comprehensive searches in these sectors that will either discover a new sector, or will rule out models that predict thermal LDM. CERN has the opportunity to play a leading role in these searches by fully exploiting the opportunities offered by the SPS and the foreseen Beam Dump Facility. 

%Answer to the big questions. 

%\begin{itemize}
%\item There is a wide mass range (roughly $10^{-12} {\rm eV} \sim 10^{13} {\rm eV} $) of the dark particle matter/dark sector, and the couplings to the SM particles are also in wide range and feebly. To cover all mass range, various efforts using the colliders, beam dumps, fixed target, optical technique are crucial. 

%\item 
%The most promising programs depend on the dark matter mass and it's properties. Many proposals in Europa covers well these parameter space, effectively and without excessive competitions.

%\item
%FCC and ILC/CLIC have excellent potentials to cover ,,,,

%\item 
%Beam dump and fixed target experiments such as SHiP, LDMX, NA62++ and NA64++, will perform the most sensitive and comprehensive searches in these sectors that will either discover a new sector, or will rule out models that predict thermal LDM. CERN has the opportunity to play a leading role in these searches by fully exploiting the opportunities offered by the SPS and the foreseen Beam Dump Facility.

%\item 
%The search for low energy particles like the axion has gained significant momentum. IAXO  provides a compelling  opportunity to discover axions. To exploit the innovative small scale experiments that could aid the future discovery as well as high precision experiments, reliable support as well as collaboration with labs on key technologies is needed.
%\end{itemize}

%\subsection{Outlook on synergies}

%{Synergy and Main discussion}
\noindent{\bf{Outlook on synergies}}: Focusing on the quest for DM in the coming decades, at the Granada Symposium
there was consensus in further developing synergies between the efforts of the high energy physics and the astrophysics communities. 
The discussion highlighted the need for enhanced communication between accelerator/collider-based, direct detection and indirect detection dark sector searches, as well as the potential benefits of common technology platforms (see Chapter~\ref{chap:inst}). 

Consensus on common search targets is important for a joint interpretation of results from different searches, and will be of fundamental importance to validate a putative DM discovery in different experiments and channels.
This can be facilitated by the existing LHC Dark Matter and Physics Beyond Collider Working Groups, and the newly established EuCAPT Astroparticle Theory Center as a joint venture of ECFA and APPEC, as well as by further discussions among the many experts in the field.

Vacuum over large volumes, cryogenics, photosensors, liquid argon detectors, design and operation of complex experiments---including software and data processing---are common themes within and beyond the communities engaged in dark matter and dark sector searches. Technological challenges related to these topics can benefit from new and existing platforms for joint discussion and collaboration.
%A common discussion of experimental issues will allow for a better overview and control of systematic uncertainties affecting multiple experiments, and will enhance the sensitivity to a broader range of dark matter and dark sector particles. 
The expertise present at CERN as the hub for the current largest collider program worldwide, together with the expertise of other large European National labs and the complementary expertise of innovative small-scale 
experiments, can stimulate knowledge transfer and add guidance and coherence to the overall DM program.


%Concrete approaches to make the collaboration are discussed and the obtained three points. A) The existing Neutrino platform are expanded to develop the common technologies for cryogenics, vacuum and photo-sensors. B) The current astroparticle programs in CERN are expand including neutrino activities. C) Theoretical collaboration working groups such as the LHC dark matter and Physics Beyond Collider and the many recognized experiments in ID, DD and accelerators. These are not limited in the dark matter session, these are useful for the Beyond Standard model session, Neutrino session and technology and computing session.
%\newpage
%~\\
%\addcontentsline{toc}{chapter}{Acknowledgements}
%{\bf{Acknowledgements}}\\
%We would like to thank the speakers of the Granada Open Symposium, 
%Hitoshi Murayama, Mariangela Lisanti, Matthew Philip Mccullough, Prateek Agrawal, Axel Lindner, Igor %Garcia Irastorza, Claudia Frugiuele, Ruth Pottgen, Elena Graverini, and Claude Vallee,
%for their thoughtful contributions; and to the many colleagues that contribute to shape the discussion %sessions during that meeting. We would also like to thank Tim Tait and Kathryn Zurek and the rest of %the members of the Physics Preparatory Group for their useful input during this writeup. 
%We would also like to thank Boyu Gao, Isabelle John, Marco Rimoldi, Antonio Boveia, Linda Carpenter, %Emma Tolley, and Francesca Ungaro for their input in producing the figures for the collider section. 
%\begin{itemize} %students & direct supervisors for plots
   % \item Boyu Gao, Isabelle John, Marco Rimoldi, Antonio Boveia, Linda %Carpenter, Emma Tolley, and Francesca Ungaro for the input on the collider %section. 
  %  \item ... for the input on the ... section. 
%\end{itemize}

% -- Add bibliography to table of contents
%\addcontentsline{toc}{chapter}{References}

% dummy reference to avoid that bibtex fails
% -- Add volume bibliography and part specific bibliographies
%%\bibliographystyle{/bib/report}
%\bibliographystyle{report}
%\bibliography{\main/bib/maindm,\main/section1/bib/section,\main/section2/bib/section,\main/section3/bi%b/section,\main/section4/bib/section}
\end{document}
